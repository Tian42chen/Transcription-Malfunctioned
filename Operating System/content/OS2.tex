\newpage
\section{Operating-System Structure}

\subsection{Operating System Services}
For user
\begin{itemize}
    \item User interface (UI): Command-Line(CLI), Graphics User Interface (GUI)
    \item Programe execution
    \item I/O operations
    \item File-system manipulation
    \item Communications 
    \item Error detection
\end{itemize}

For system
\begin{itemize}
    \item Resource allocation
    \item Accounting
    \item Protection and security
\end{itemize}

\subsection{System Call}
使用高级语言写 system call. 

Application Program Interface (API) 与 system call 看作是不同的, API 层级更高. 

POSIX API 标准化了 API.

\subsubsection{Implementation}
每个 system call 会有一个 number. System-call interface 维护一个 number 索引的 table. 

封装 好处: 抽象, 可移植 etc.

\begin{figure}[!htb]
    \centering
    \includegraphics[width=0.42\textwidth]{pic/OS2/API – System Call – OS Relationship}
    \caption{API – System Call – OS Relationship}
\end{figure}

\subsubsection{Parameter Passing}
Three general methods used to pass parameters to the OS: 
\begin{itemize}
    \item Simplest: pass the parameters in \textbf{registers}. 但有时不够用. 
    \item Parameters stored in a \textbf{block}, or table, in memory, and address of block passed as a parameter in a register. 
    \item Parameters placed, or pushed, onto the \textbf{stack} by the program and popped off the stack by the operating system. 
\end{itemize}
Block and stack methods do not limit the number or length of parameters being passed. 


\subsection{Types of System Calls}

\begin{itemize}
    \item Process control
    \item File management
    \item Device management
    \item Information maintenance (e.g. time, date)
    \item Communications
    \item Protection    
\end{itemize}


\subsection{System Programs}
System programs provide a convenient environment for program development and execution. They can be divided into:
\begin{itemize}
    \item File manipulation
    \item Status information
    \item File modification
    \item Programming language support
    \item Program loading and execution
    \item Communications
    \item Application programs
\end{itemize}


\subsection{Operating System Design and Implementation}
OS设计与实现没有明确输入输出. Start by defining goals and specifications. Affected by choice of hardware, type of system. 

\begin{itemize}
    \item User goals --- operating system should be convenient to use, easy to learn, reliable, safe, and fast
    \item System goals --- operating system should be easy to design, implement, and maintain, as well as flexible, reliable, error-free, and efficient
\end{itemize}


\subsubsection{Mechanism and Policy} %TODO 读 2.7.2 P80
Important principle to separate: (分离策略与机制)
\begin{itemize}
    \item Policy: What will be done? 策略(确定具体做什么事)
    \item Mechanism: How to do it? 机制(定义做事方式)
\end{itemize}

\subsection{Operating System Structure}

\subsubsection{Layered Approach}
仅帮助人类理解. 
\begin{figure}[!htb]
    \centering
    \includegraphics[width=0.309\textwidth]{pic/OS2/Layered Operating System}
    \caption{Layered Operating System}
\end{figure}

\subsubsection{UNIX}
Monolithic structure (宏内核) 
\begin{figure}[!htb]
    \centering
    \includegraphics[width=0.42\textwidth]{pic/OS2/Traditional UNIX system structure}
    \caption{Traditional UNIX system structure}
\end{figure}

\subsubsection{Microkernel System Structure}
微内核. 把很多操作分到 user code 之中. 
Benefits: 代码少, 稳定, 易移植. 
Detriments: 效率变低 (Performance overhead). 


\begin{figure}[!htb]
    \centering
    \includegraphics[width=0.42\textwidth]{pic/OS2/Architecture of A Typical Microkernel}
    \caption{Architecture of A Typical Microkernel}
\end{figure}




% \subsection{Virtual Machines}
% \subsection{Operating System Generation}
% \subsection{System Boot}



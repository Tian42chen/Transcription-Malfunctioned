\newpage
\section{Processes}
\subsection{Process Concept}
\subsubsection{Concept}
又称 jobs/tasks/process. Process是一个运行的实例. 

A process includes:
\begin{itemize}
    \item text section (code, 代码段)
    \item data section (global vars, 数据段)
    \item stack (function parameter, local vars, return addresses, 栈)
    \item heap (dynamically allocated memory, 堆)
\end{itemize}

heap增长需要 OS 操作, stack 是已分配的一块空间, 在其内增长, 超过触发 stack overflow. stack 与 heap 之间的区域称为 hole, 占据超过 90\% 的空间. 

\begin{figure}[!htb]
    \centering
    \includegraphics[width=0.22\textwidth]{pic/OS3/Layout of a process in memory.png}
    \caption{Layout of a process in memory}
\end{figure}

kernel stack 与 程序的 stack 不是一个. 每个 process 对应一个 kernel stack. 

\subsubsection{Process State}
As a process executes, it changes state:
\begin{itemize}\small
    \item new: The process is being created
    \item running: Instructions are being executed
    \item waiting: The process is waiting/blocked for some event to occur
    \item ready: The process is waiting to be assigned to a processor
    \item terminated: The process has finished execution
\end{itemize}

\begin{figure}[!htb]
    \centering
    \includegraphics[width=0.42\textwidth]{pic/OS3/Diagram of process state.png}
    \caption{Diagram of process state transition}
\end{figure}
即使cpu空闲仍会有 idle 进程 一直在运行, 所以无论如何都需要 ready 缓冲. 

\subsubsection{Process Control Block (PCB)}
Information associated with each process: 
\begin{itemize}
    \item Process state
    \item Program counter
    \item Contents of CPU registers
    \item CPU scheduling information
    \item Memory-management information
    \item Accounting information
    \item I/O status information
\end{itemize}
这些由PCB存储.

\begin{figure}[!htb]
    \centering
    \includegraphics[width=0.42\textwidth]{pic/OS3/Diagram showing context switch from process to process}
    \caption{Diagram showing context switch from process to process}
\end{figure}
二者idle 相交称为 overhead, 是为了 multiprocessing 留的冗余. 

\subsection{Process Scheduling}
\begin{itemize}
    \item Job queue --- set of all processes in the system
    \item Ready queue --- set of all processes residing in main memory, ready and waiting to execute
    \item Device queues --- set of processes waiting for an I/O device
\end{itemize}
Processes migrate(迁移) among the various queues. 一个 process 仅能在一个 queue 中. 

\subsubsection{Schedulers}
\begin{itemize}
    \item Long-term scheduler (or job scheduler) (过时了, 现在由用户决定, 是 user scheduler) --- selects which processes should be brought into memory (the ready queue). 控制着 the degree of multiprogramming. 
    \item Short-term scheduler (or CPU scheduler) --- selects which process should be executed next and allocates CPU. 频繁触发, 为了保证系统的交互性. 
\end{itemize}
UNIX and Windows do not use long-term scheduling

\paragraph{Medium Term Scheduling} 需要时将 PCB 写入磁盘 (swap out), 然后合适时恢复PCB到ready queue (swap in). 

Processes can be described as either:
\begin{itemize}
    \item I/O-bound process (IO绑定进程) --- spends more time doing I/O than
    computations, many short CPU bursts
    \item CPU-bound process (CPU绑定进程) --- spends more time doing
    computations; few very long CPU bursts
\end{itemize}

\subsubsection{Context Switch}
当切换进程时, 需要保存当前进程的上下文, 并加载切换进程的上下文. swtch 是特殊的命令, 不是拼写错误.  %TODO read xv6, 7.1

\begin{figure}[!htb]
    \centering
    \includegraphics[width=0.42\textwidth]{pic/oS3/context switch.png}
    \caption{Switching from one user process to another. In this example, xv6 runs with one CPU (and thus one scheduler thread).}
\end{figure}

\subsection{Operations on Processes}
\subsubsection{Process Creation}
Parent process creates children processes, which, in turn create other processes, forming a tree of processes. Parent and children share all resources. Child duplicate of parent. 

使用 fork() 产生子进程, 基本相当于复制进程. 

wait() 等待子进程的完成, 会返回信息.

exec() 运行另外的程序, 抹杀当前的程序. 

\begin{figure}[!htb]
    \centering
    \includegraphics[width=0.42\textwidth]{pic/OS3/fork.png}
    \caption{Process creation using the fork() system call}
\end{figure}

\subsubsection{Process Termination}
exit() 释放进程占用资源, 并通知子进程, the child gets orphaned and its parent becomes the ``init'' process (PID=1). 

abort()

\subsection{Cooperating Processes}
有 Independent process 与 Cooperating process. 

Advantages of process cooperation: 
\begin{itemize}
    \item Information sharing
    \item Computation speed-up (Multiple CPUs)
    \item Modularity
    \item Convenience
\end{itemize}

\subsubsection{Producer-Consumer Problem}
经典的合作模型, 可描述调用关系. 生产者产生信息, 消费者消费信息. 有两种:
\begin{itemize}
    \item unbounded-buffer: 信息相比buffer微不足道. 
    \item bounded-buffer: 信息相比buffer不能忽略. 
\end{itemize}

以 Bounded-Buffer 为例:
\begin{itemize}
    \item Shared-Memory Solution: Shared data. 共享数据. 
    \begin{minted}{c++}
        #define BUFFER_SIZE 10
        typedef struct {
            . . .
        } item;
        item buffer[BUFFER_SIZE];
        int in = 0;
        int out = 0;
    \end{minted}
    \item Insert() Method
    \begin{minted}{c++}
        while (true) {
            Produce an item;
            while ((in + 1) % BUFFER_SIZE == out) 
                ; /* do nothing -- no free buffers */
            buffer[in] = item;
            in = (in + 1) % BUFFER_SIZE;
        }
    \end{minted}
    \item Remove() Method
    \begin{minted}{c++}
        while (true) {
            while (in == out)
                ; //do nothing, nothing to consume
            Remove an item from the buffer;
            item = buffer[out];
            out = (out + 1) % BUFFER SIZE;
            return item;
        }
    \end{minted}
\end{itemize}

\subsection{Interprocess Communication (IPC)}
进程间进行通讯. 

Two models for IPC: message passing and shared memory. 

Message-passing facility provides two operations:
\begin{itemize}
    \item send(message) – message size fixed or variable
    \item receive(message)
\end{itemize}

\begin{figure}[!htb]
    \centering
    \includegraphics[width=0.309\textwidth]{pic/OS3/Communications models}
    \caption{Communications models. (a) Shared memory. (b) Message passing.}
\end{figure}

\subsubsection{Direct Communication}
Processes must name each other explicitly:
\begin{itemize}\small
    \item \textbf{send} (P, message) – send a message to process P
    \item \textbf{receive}(Q, message) – receive a message from process Q
\end{itemize}

\subsubsection{Indirect Communication}
Messages are directed and received from \textbf{mailboxes} (also referred to as \textbf{ports})
\begin{itemize}
    \item Each mailbox has a unique id (well-known ID)
    \item Processes can communicate only if they share a mailbox
\end{itemize}

Operations:
\begin{itemize}
    \item create a new mailbox
    \item send and receive messages through mailbox
    \item destroy a mailbox
\end{itemize}

Primitives are defined as:
\begin{itemize}\small
    \item \textbf{send} (A, message) – send a message to mailbox A
    \item \textbf{receive}(A, message) – receive a message from mailbox A
\end{itemize}

\subsubsection{Synchronization}
Message passing may be either blocking or non-blocking.

Blocking is considered synchronous

Non-blocking is considered asynchronous

\subsubsection{Buffering}
Queue of messages attached to the link; implemented in one of three ways
\begin{enumerate}
    \item Zero capacity – 0 messages
    \subitem Sender must wait for receiver
    \item Bounded capacity – finite length of n messages
    \subitem Sender must wait if link full
    \item Unbounded capacity – infinite length
    \subitem Sender never waits
\end{enumerate}

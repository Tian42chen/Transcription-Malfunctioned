\section[Relations]{Relations (关系)}

\subsection{Relations and Their Properties}

\subsubsection{Relations}
\begin{definition}
    A \highlight{binary relation} $R$ between $A$ and $B$ is a subset of Cartesian product $A\times B$
    \[ R\subseteq A\times B \]
    when $A=B$, $R$ is called a relation on set $A$. 
\end{definition}

\begin{enumerate}
    \item Given a relation $R$ from $A$ to $B$. 
    \begin{itemize}
        \item The \highlight{domian} of $R$
        \[ Dom(R)=\left\{ x\in A |\, \exists y \in B, \, (x,y) \in R  \right\} \]
        \item The \highlight{range} of $R$
        \[ Ram(R)=\left\{ y\in B | \,\exists x\in A ,\, (x,y)\in R \right\} \]
    \end{itemize}
    \item Function as relations: Recall a function $f$ from set $A$ to $B$. 
    \item n-ary relations: Let $A_1,A_2,\dots,A_n$ be sets. An n-ary relation on these sets is a subset of $A_1\times A_2\times \cdots\times A_n$. $n$  is called the degree. 
\end{enumerate}

\subsubsection{Combining Relations}
\begin{enumerate}
    \item Union, intersection, complement and difference of relations (并交补差). 
    \item Let $R$ be a relation from a set $A$ to $B$ and $S$ a relation from $B$ to $C$. The \highlight{composite (合成)} of $R$ and $S$ is the relation
    \begin{align*}
        S \circ R =\left\{ (a,c)|\, a\in A,c\in C \,\exists b\in B ,\, \begin{array}{l}
            (a,b)\in R\\(b,c)\in S
        \end{array}  \right\}
    \end{align*}
    \item LEt $R$ be a relation on set $A$. The \highlight{power (幂) $R^n$} $n=1,2,\dots$ are defined inductively by 
    \begin{align*}
        R^1=&R \\
        R^{n+1}=&R^n \circ R
    \end{align*}
    \item Given a relation $R$ from $A$ to $B$, its \highlight{inverse $R^{-1}$} is the relation from $B$ to $A$ defined by
    \begin{align*}
        R^{-1}=\left\{ (y,x)| (x,y)\in R \right\}
    \end{align*}
\end{enumerate}

\quad

\subsubsection{Properties of Relations}
\begin{definition}
    Let $R$ be a relation on a set $A$. 
    \begin{align*}
        R \text{ is \highlight{reflexive (自反)}}\Longleftrightarrow \forall x \in A, (x,x)\in R\\
        R \text{ is \highlight{irreflexive (反自反)}}\Longleftrightarrow \forall x \in A, (x,x)\notin  R
    \end{align*}
\end{definition}
\begin{definition}
    Let $R$ be a relation on a set $A$. 
    \begin{align*}
        R \text{ is \highlight{symmetric (对称)}}\Longleftrightarrow& \forall x,y \in A, (x,y)\in R\\& \Rightarrow (y,x)\in R\\
        R \text{ is \highlight{anti-symmetric (反对称)}}\Longleftrightarrow& \forall x,y \in A, (x,y)\in R \\ &\text{and } (y,x)\in R \Rightarrow x=y
    \end{align*}
\end{definition}
Remark:
\begin{enumerate}
    \item $R \text{ is symmetric} \Longleftrightarrow R^{-1}=R$
    \item $R \text{ is anti-symmetric} \Longleftrightarrow R\cap R^{-1}\subseteq R_=$
    \item Non-symmetric $\nLeftrightarrow $ anti-symmetric
\end{enumerate}
\begin{definition}
    Let $R$ be a relation on a set $A$. 
    \begin{align*}
        R \text{ is \highlight{transitive (传递)}}\Longleftrightarrow& \forall x,y,z\in A\\ & \left( (x,y)\in R \land (y,z)\in R \right) \\ &\Rightarrow (x,z)\in R
    \end{align*}
\end{definition}
Remark: $R$ is transitive $\Leftrightarrow R\circ R \subseteq R$. 

\begin{theorem}
    The relation $R$ on a set $A$ is transitive iff 
    \[ R^n \subseteq R \]
    for $n=2,3,\dots$
\end{theorem}

\subsection{Representing Relations}
\subsubsection{Matrices of Relations}
\begin{definition}
    Suppose $R$ a relation \\ from $A=\left\{ a_1,a_2,\dots,a_m \right\}$ to $B=\left\{ b_1,b_2,\dots,b_n \right\}$, The relation $R$ can be represented by matrix $M_R=(m_{ij})_{m\times n}$
    \[ m_{ij}=\left\{ \begin{array}{lc}
        1 & (a_i,b_j)\in R \\ 0 & (a_i,b_j)\notin R
    \end{array} \right. \]
\end{definition}
Remark:
\begin{enumerate}
    \item Let $M_R$ be the matrix of a relation $R$ on set $A$. Let $M_R^2=M_R\circ M_R$.
    \begin{enumerate}
        \item $R$ is reflexive $\Leftrightarrow$ $m_{ii}=1$.\\
        $R$ is irreflexive $\Leftrightarrow$ $m_{ii}=0$.
        \item  $R$ is symmetric $\Leftrightarrow$ $m_{ij}=m_{ji}$, i.e. $M_R$ is a symmetric matrix. \\
        $R$ is anti-symmetric $\Leftrightarrow$ $m_{ij}=1, i\ne j \Rightarrow m_{ji}=0$. 
        \item $R$ is transitive $\Leftrightarrow$ whenever $c_{ij}$ in $C=M_R^2$ is nonzero then entry $m_{ij}$ in $M_R$ is also nonzero. 
        \begin{align*}
            c_{ij}=a_{i1}a_{1j}\lor a_{i2}a_{2j} \lor \cdots \lor a_{ik}a_{kj} \lor \cdots \lor a_{in}a_{nj}
        \end{align*}
    \end{enumerate}
    \item Suppose that $R_1$ and $R_2$ are relations on a set $A$ represented by matrix $M_{R_1}$ and $M_{R_2}$ respectively. Then 
    \begin{align*}
        M_{R_1\cup R_2}=&M_{R_1}\lor M_{R_2}\\
        M_{R_1\cap R_2}=&M_{R_1}\land M_{R_2}\\
        M_{S\circ R}=M_R\odot  M_S
    \end{align*}
    where the operator are join and meet. 
\end{enumerate}

\subsubsection{Digraphs of Relations}
Each element of the set is represented by a point, and each ordered pair is represented using an arc with its direction indicated by an arrow --- \highlight{directed graphs} or digraphs (元素为结点, 关系为有向边).

Remark: Let $R$ be a relation on set $A$. 
\begin{enumerate}
    \item $R$ is reflexive $\Leftrightarrow$ There are loops at every vertex of digraph. 
    \item $R$ is symmetric $\Leftrightarrow$ Every edge is Bi-directional edge. 
\end{enumerate}

\subsection{Closures of Relations}
\subsubsection{Introduction}
\begin{definition}
    Let $R$ be a relation on a set $A$. If there is a relation $S$ satisfy:
    \begin{enumerate}
        \item $S$ with property $P$ (reflexive, symmetricm, or transitive) and $R\subseteq S$. 
        \item $\forall S'$ with property $P$ and $R\subseteq S'$, then $S\subseteq S'$.
    \end{enumerate}
    Then $S$ is called the \highlight{closure} of $R$ with respect to $P$. 
\end{definition}

\subsubsection{Computing of Closures}
\begin{theorem}
    Let $R$ be a relation on set $A$. 
    \begin{enumerate}
        \item The \highlight{reflexive closure} of relation $R$:
        \begin{align*}
            r(R)=R\cup \Delta
        \end{align*}
        where $\Delta=\left\{ (a,a) | a\in A \right\}$ is diagonal relation on $A$. 
        \item The \highlight{symmetric closure} of relation $R$: 
        \begin{align*}
            s(R)=R\cup R^{-1}
        \end{align*}
        \item The \highlight{transitive closure} of $R$:
        \begin{align*}
            t(R)=R^*
        \end{align*}
    \end{enumerate}
\end{theorem}

\begin{definition}
    A \highlight{path} from $a$ to $b$ in the digraph $G$ is a sequence of one or more edges $(x_0,x_1)$, $(x_1,x_2)$, $\dots$, $(x_{n-1},x_n)$ in $G$. where $x_0=a$ and $x_n=b$. The path is denoted by $x_0,x_1,\dots,x_n$ and has length $n$. 
    \begin{itemize}
        \item A \highlight{circuit} or \highlight{cycle}: a path that begins and ends at the same vertex. 
    \end{itemize}
\end{definition}

\begin{theorem}
    Let $R$ be a relation on set $A$. There is a path of length $n$ from $a$ to $b$ $\Leftrightarrow$ $(a,b)\in R^n$.
\end{theorem}

\begin{definition}
    The \highlight{connectivity relation}\\ $R^*=\left\{ (a,b)| \text{there is a path from $a$ to $b$} \right\}$. 
    \begin{align*}
        R^*=\bigcup_{n=1}^{\infty}R^n
    \end{align*}
\end{definition}

\begin{theorem}
    The transitive closure of a relation $R$ equals the connectivity relation $R^*$, i.e.
    \begin{align*}
        t(R)=R^*
    \end{align*}
\end{theorem}

Let $R$ be a relation on set $A$ with $n$ elements. Then,
\begin{align*}
    R^*=\bigcup_{i=1}^n R^i
\end{align*}

\begin{theorem}
    Let $M_R$ be the zero-one matrix of the relation $R$ on a set with $n$ elements. Then the zero-one matrix of the transitive closure $R^*$ is 
    \begin{align*}
        M_{R^*}=M_R\lor \cdots \lor M_R^{[n]}
    \end{align*}
\end{theorem}

$O(n^4)$

\subsubsection{Warshall's Algorithm}
$O(2n^3)$

\subsection{Equivalence Relations}
\subsubsection{Introduction of Equivalence Relations}
\begin{definition}
    Relation $R_{\sim}: A\leftrightarrow  A$ is an \highlight{equivalence relation (等价关系)}, if it reflexive, symmetric and transitive. 
\end{definition}

\subsubsection{Equivalence Classes and its Properties}
\begin{definition}
    Let $R: A \leftrightarrow A$ is an equivalence relation. $\forall a\in A$, the \highlight{equivalence class (等价类)} of $a$ is the set of the elements related to $a$
    \[ [a]_R=\left\{ x\in A | (x,a)\in R \right\} \]
    If $b\in [a]_R$, $b$ is called a representative of this equivalence class. 
\end{definition}

The properties of equivalence classes:
\begin{enumerate}
    \item $\forall a\in A, [a]_R\ne \emptyset $.
    \item $(a,b)\in R \Rightarrow [a]_R=[b]_R$
    \item $(a,b)\notin R \Rightarrow [a]_R\cap[b]_R=\emptyset$
    \item $\displaystyle \bigcup_{a\in A} [a]_R=A$
\end{enumerate}

\begin{definition}
    The set of all equivalence classes of $R$ is the \highlight{quotient set (商集)} of $A$ with respect to $R$
    \begin{align*}
        \frac{A}{R}=\left\{ [a]_R|a\in A \right\}
    \end{align*}
\end{definition}
Remark:
\begin{enumerate}
    \item If $A$ finite, then $\frac{A}{R}$ finite. 
    \item If $A$ has $n$ elements, and if every $[a]_R$ has $m$ elements, then $m|n$, and $\frac{A}{R}$ has $\frac{n}{m}$ elements. 
\end{enumerate}

\subsubsection{Partition}
\begin{definition}
    A \highlight{partition (划分)} $\pi$ on a set $S$ is a family $\left\{ A_1,A_2,\dots,A_n \right\}$ of subsets of $S$ and 
    \begin{enumerate}
        \item $\displaystyle \bigcup_{k=1}^n A_k=S$.
        \item $\displaystyle A_j\cap A_k=\emptyset$ for every $j,k$ with $j\ne k,\, 1<j,k<n$. 
    \end{enumerate}
\end{definition}
\begin{theorem}
    Let $R$ be an equivalence relation on a set $S$. Then the equivalence classes of $R$ form a partition of $X$. Conversely, given a partition $\left\{A_i|i\in I\right\}$ of set $S$, there is an equivalence relation $R$ that has the set $A_i,i\in I$, as its equivalence classes. 
\end{theorem}

\subsection{Partial Orderings}
\subsubsection{Introduction}
\begin{definition}
    Relation $R_{\preceq }: S\leftrightarrow S$ is a \highlight{partial order}, if it reflexive, anti-symmetric and transitive. 

    A set $S$ together with a partial ordering $R_{\preceq}$ is called a \highlight{partial order set} or \highlight{poset}, and is denoted by $(S,R_{\preceq}), P(S)$. 
\end{definition}

\begin{definition}
    $\forall a,b\in poset(S,\preceq)$ are called \highlight{comparable (可比)} if either $a\preceq b$ or $b\preceq a$, otherwise they are called \highlight{incomparable (不可比)}.
    
    If $(S,\preceq)$ is a poset and every elements of $S$ are comparable, $S$ is called a \highlight{totally order (全序集)} or \highlight{linearly order set (线性集)}, $\preceq$ is called a \highlight{total order or linear order (全序或线性)}. A totally ordered set is also called a \highlight{chain (链)}. 
\end{definition}

\subsubsection{Lexicographic Order}
\subsubsection{Hasse Diagram}
Represent a partial ordering on a finite set using the following procedure: 
\begin{enumerate}
    \item Start with the directed graph for the relation. 
    \item Remove all loops. 
    \item Remove all edges that must be present because of the transitivity. 
    \item Finally, arrange each edges so that its initial vertex is  below its terminal vertex, remove all the arrows. 
\end{enumerate}
The resulting diagram contains sufficient information to find the partial ordering. --- Hasse Diagram. 

\subsubsection{Maximal and Minimal Elements}
\begin{definition}
    Let $(A,\preceq)$ be a partial ordered set, $B\subseteq A$.
    \begin{enumerate}
        \item maximal and minimal elements
        \begin{enumerate}
            \item $a$ is a \highlight{maximal element (极大元)} of $B$: 
            \begin{align*}
                a\in B \land \nexists\, x\in B:\, a\prec x
            \end{align*}
            \item $b$ is a \highlight{minimal element (极小元)} of $B$: 
            \begin{align*}
                b\in B \land \nexists\, x\in B:\, x\prec b
            \end{align*}
        \end{enumerate}
        \item greatest and least elements
        \begin{enumerate}
            \item $a$ is the \highlight{greatest element (最大元)} of $B$: 
            \begin{align*}
                a\in B \land \forall\, x\in B:\, x\preceq a
            \end{align*}
            \item $b$ is the \highlight{least element (最小元)} of $B$: 
            \begin{align*}
                b\in B \land \forall\, x\in B:\, b\preceq x
            \end{align*}
        \end{enumerate}
        \item upper and lower bound
        \begin{enumerate}
            \item c is an \highlight{upper bound (上界)} of $B$:
            \begin{align*}
                c\in A \land \forall\, x \in B:\, x\preceq c
            \end{align*}
            \item d is a \highlight{lower bound (下界)} of $B$: 
            \begin{align*}
                d\in A \land \forall\, x \in B:\, d\preceq x
            \end{align*}
        \end{enumerate}
        \item least upper and greatest lower bound
        \begin{enumerate}
            \item c is the \highlight{least upper bound (最小上界)} of $B$:
            \begin{align*}
                c\text{ is an upper bound of }B&\\ \land \forall\, x \text{ is an upper bound of }B&:\, c\preceq x
            \end{align*}
            \item d is the \highlight{greatest lower bound (最大下界)} of $B$: 
            \begin{align*}
                d\text{ is an lower bound of }B&\\ \land \forall\, x \text{ is an lower bound of }B&:\, x\preceq d
            \end{align*}
        \end{enumerate}
    \end{enumerate}
    $(S,\preceq)$ is a well-ordered set if it is a poset such that $\preceq$ is a total ordering and such that every nonempty of $S$ has a least element. 
\end{definition}

\subsubsection{Lattices}
\begin{definition}
    A partially ordered set in which pair of elements has both a least upper bound and a greatest lower bound is called \highlight{lattice (格)}.
\end{definition}
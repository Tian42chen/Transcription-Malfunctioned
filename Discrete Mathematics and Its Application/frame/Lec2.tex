\section[Basic Structures: Sets, and Functions]{Basic Structures: Sets, and Functions (集合与函数)}

\subsection{Sets}

\subsubsection{Properties of sets}
\begin{itemize}
    \item Order of elements doesn't matter.
    \item Repetition of elements doesn't matter. 
    \item Certainty. 
\end{itemize}

\subsubsection{Infinite and Finite Set}
Cardinality of set $S$ ($\left| S \right|$) is the number of elements in $S$. 
\begin{itemize}
    \item Infinite Countable.
    \item Uncountable. 
\end{itemize}

\subsubsection{Subsets}
\begin{enumerate}
    \item Subset notation: $\subseteq$  
    \begin{align*}
        S \subseteq T \Longleftrightarrow \forall x \in S \longrightarrow x\in T
    \end{align*}
    \item Proper Subset: $\subset $
    \begin{align*}
        S \subset T \Longleftrightarrow \forall x \in S \longrightarrow x\in T \text{ and } S\neq T
    \end{align*}
    \item Empty set $\emptyset $ and Universal set $U$:
    
    For any set A:
    \begin{align*}
        A \subseteq A\\
        \emptyset \subseteq A \subseteq U
    \end{align*}
\end{enumerate}

\subsection{Set Operations}
\begin{enumerate}
    \item Union
    \begin{align*}
        A \cup B = \left\{ x | x \in A \lor x \in B  \right\}
    \end{align*}
    \item Intersection
    \begin{align*}
        A \cap  B = \left\{ x | x \in A \land x \in B  \right\}
    \end{align*}
    \item Difference
    \begin{align*}
        A-B = \left\{ x| x\in A \lor x \notin B \right\}
    \end{align*}
    \item Complement
    
    Let $U$ be the universal set. 
    \begin{align*}
        \bar{A}=U-A
    \end{align*}
    \item Symmetric Difference
    \begin{align*}
        A \oplus B =(A-B)\cup (B-A)
    \end{align*}
\end{enumerate}

\subsubsection{The Power Set}
\begin{definition}
    Given s set S, \textcolor{light_red}{the power set} of S is the set of all subsets of the set S. The power set of S is denoted by P(S) or $2^S$.
    \[ 2^S = \left\{ T | T \subseteq S  \right\} \]
\end{definition}

Remark: If $|S|=n$, $|2^S|=2^n$. 
\subsubsection{Cartesian Products}
\begin{definition}
    \textcolor{light_red}{The ordered n- tuple} $(a_1,a_2,\cdots,a_n)$ is the ordered collection that has $a_1$ as its first element, $\cdots$, and $a_n$ as its $n$-th element. 
\end{definition}
\begin{itemize}
    \item $(a_1,\cdots, a_n)=(b_1,\cdots,b_n)\Longleftrightarrow a_i=b_i \text{ for } i=1,\cdots, n$.
    \item In particular, 2-tuples are called ordered pairs (序偶). 
\end{itemize}

\begin{definition}
    Let A and B be sets. \textcolor{light_red}{The Cartesian product of A and B} , denoted by A$\times$B, is the set of all ordered pairs (a,b), where $a\in A$ and $b\in B$. Hence,
    \[ A\times B = \left\{ (a,b) | a\in A \land b \in B  \right\} \]
\end{definition}

\begin{itemize}
    \item $A^2=A\times A$ the Cartesian square of A. 
    \item If $A$ and $B$ finite, $A \times B$ finite.
    
    If $A$ has $m$, $B$ has $n$, $A \times B$ has $mn$.
    \item If $A$ infinite and $B$ non-empty, $A\times B, B\times A$ infinite.
\end{itemize}

Properites of Cartesian Products: 
\begin{enumerate}
    \item $A \times \emptyset=\emptyset \times B =\emptyset$.
    \item In general, $A \times B\neq B\times A$.
    \item In general, $(A \times B)\times C \neq A \times (B\times C)$, unless we identify $((a,b),c)$ and $(a,(b,c))$. 
    \item \begin{align*}
        A \times (B \cup C) &= (A \times B )\cup (A \times C)\\
        A \times (B \cap C) &= (A \times B )\cap (A \times C)\\
    \end{align*}
\end{enumerate}

\begin{definition}
    \textcolor{light_red}{The Cartesian porduct of $A_1, A_2, \cdots, A_n$}, denoted by $A_1\times A_2\times \cdots \times A_n$, is the set of all ordered n-tuples $(a_1,a_2,\cdots, a_n)$, where $a_i \in A_i$ for $i=1,2,\cdots, n$. In other words, 
    \[ A_1\times \cdots \times A_n=\left\{ (a_1,\cdots, a_n)| a_i \in A_i \text{ for } i=1,\cdots, n \right\} \]
\end{definition}

\begin{itemize}
    \item $A^n=A\times A \times \cdots \times A$($n$ times).
    \item \begin{align*}
        (A \cap B )\times (C \cap D)=(A\times C)\cap (B\times D)\\
        (A \cup B )\times (C \cup D)\ne (A\times C)\cup (B\times D)\\
    \end{align*}
\end{itemize}

\subsection{Cardinality of Finite and Infinite Sets}

\subsubsection{Counting Finite Sets}

\begin{enumerate}
    \item Cardinality: $|S|$.
    \item Prnciple of Inclusion-exclutsion:
    \begin{align*}
        |A\cup B|=& |A|+|B|-|A\cap B|\\
        \left|\bigcup_{i=1}^n A_i\right|=&\sum_{i=1}^n \left| A_i \right|-\sum_{1\le i \ne j \le n } \left| A_i\cap A_j \right| + \cdots \\
        &+ (-1)^{n-1} \left| \bigcap_{i=1}^n A_i \right|
    \end{align*}
\end{enumerate}

\subsubsection{Cardinality of Infinite Sets}
$|A|=|B|$, $A,B$ have the same cardinality (equinumerous), iff there is an one-to-one correspondence (双射) from $A$ to $B$.

Remark: 
\begin{enumerate}
    \item Two sets have the same cardinality is a equivalence relation (等价关系). 
    \item Two sets have the same cardinality, but the one-to-one correspondence may be not unique. 
    \item A set and its proper set have the same cardinality iff it is infinite set. 
\end{enumerate}

\subsubsection{Two Types of Infinite sets}


\begin{itemize}
    \item Countable (denumberable) set ($\mathbb{N}$) is either finite or has the same cardinality as the set of natural numbers $\mathbb{N}$, $\aleph_0$ is called countable. 
    \item  And other are  uncountable set. 
\end{itemize}


Some special infinite sets: 
\begin{enumerate}
    \item The set of integers is countable, $|\mathbb{N}|=|\mathbb{Z}|$.
    \item The set of rational number is countable, $|\mathbb{N}|=|\mathbb{Q}|$. 
    \item The set of real nnumbers is uncountable, $|R|=\aleph >\aleph_0 $.
    \item The set $\mathbb{N}\times \mathbb{N}$ is uncountable. 
    \item The uncountable set always has a proper set that is countable. 
\end{enumerate}

\begin{theorem}[Cantor Theorem]
    The cardinality of the power set of an arbitrary set has a greater cardinality than the original arbitrary set, or 
    \[ \left|2^A\right| > |A| \].
\end{theorem}

\subsection{Functions}

\subsubsection{Introduction}
\begin{definition}
    Let A and B be sets, a \textcolor{light_red}{function f} from A to B: 
    \[
        f : A \longrightarrow B \Longleftrightarrow \forall a \in A \, \exists !b \in B (b unique): f(a)=b
    \]
    f maps A to B. 
    \begin{itemize}
        \item A is \textcolor{light_red}{the domain} of f.
        \item B is \textcolor{light_red}{the codomain} of f.
        \item $f(a)=b, a\in A, b \in B$, b is the \textcolor{light_red}{image} of a, a is a \textcolor{light_red}{pre-image} of b. 
        \item The \textcolor{light_red}{range} of f is the set: 
        \[ \mathrm{Range}(f)=\left\{ b \in B | \exists a \in A , f(a)=b \right\} \]
    \end{itemize}
\end{definition}

\subsubsection{One-to-one and Onto Functions}
\begin{definition}
    Lef f be a function fro A to B: 
    \begin{itemize}
        \item \textcolor{light_red}{one-to-one function}: (injective) (单射)
        \[ \forall a, b \in A \land a \ne b \Longrightarrow f(a \ne f(b)) \]
        \item \textcolor{light_red}{onto function}: (surjective) (满射)
        \[ \forall b \in B \, \exists a \in A \text{ such that } f(a)=b \]
        \item \textcolor{light_red}{bijection function}:  (one-to-one correspondence) (双射)
        
        \[\text{one-to-one} + \text{onto}\]
    \end{itemize}
\end{definition}

\subsubsection{Inverse and composition of function}
\begin{definition}
    Let f be a one-to-one correspondence from A to B. \textcolor{light_red}{The inverse function} of f, $f^{-1}: B \longrightarrow A$ is 
    \[ \forall a \in A ,b\in B (f(a)=b) \Longleftrightarrow (f^{-1}(b)=a) \]
\end{definition}

Remark: 
\begin{enumerate}
    \item A one-to-one correspondence is called invertible.
    \item A function is not invertible if it is not a one-to-one correspondence. 
\end{enumerate}

\begin{definition}
    Let $g: A \longrightarrow B$ and $f: B \longrightarrow C$ are two functions. \textcolor{light_red}{The composition} of the functions f and g, $f\circ g: A \longrightarrow C$ is 
    \[ \forall a \in A, (f\circ g)(a)=f(g(a)) \]
\end{definition}

\subsubsection{Some Important Functions}
\begin{definition}
    The floor functions and the ceiling function.
    \begin{itemize}
        \item \highlight{The floor functions} $\lfloor x \rfloor$ assigns the real number x the largest integer that $\le$ x.
        \item \highlight{The ceiling function} $\lceil x \rceil$ assigns to the real number x the smallest integer that is $\ge$ x. 
    \end{itemize}
\end{definition}

Remark: 
\begin{enumerate}
    \item $x-1<\lfloor x \rfloor\le x \le \lceil x \rceil < x+1$
    \item $\lceil -x \rceil=-\lfloor x \rfloor$
    \item $\lfloor -x \rfloor=-\lceil x \rceil$
\end{enumerate}

\subsubsection{The Growth of Functions}
\begin{definition}[Big-O, Big-Omega, Big-Theta]
    Let f and g be functions from the set of integers or the set of  real numbers. 
    \begin{itemize}
        \item We say that f(x) is \highlight{O(g(x))} if there exist constants C and k such that 
        \[ |f(x)\le C|g(x)||\]
        where x>k. 
        \item We say that f(x) is \highlight{$\Omega(g(x))$} if there exist constants C and k such that
        \[ |f(x)|\ge C|g(x)| \]
        where x>k.
        \item We say that f(x) is \highlight{$\Theta (g(x))$} if f(x)=O(g(x)) and $f(x)=\Omega (g(x))$. We also say that f(x) is of \highlight{order} g(x). 
    \end{itemize}
\end{definition}

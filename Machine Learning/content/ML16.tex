\newpage
\section{强化学习}
\subsection{任务与奖赏}
强化学习常用马尔可夫决策过程 (Markov Decision Process, MDP) 描述. 

强化学习对应了四元组
\begin{align*}
    E=\braket{X, A, P, R}
\end{align*}
其中 $P$ 指定了状态转移概率, $R$ 指定了奖赏, $X$ 为状态空间, $A$ 为动作空间. $E$ 为机器所处环境. 

强化学习的目标: 机器通过在环境中不断尝试从而学到一个策略$\pi$,使得长期执行该策略后得到的累积奖赏最大


强化学习在某种意义上可以认为是具有“延迟标记信息”的监督学习. 

\subsection{K-摇臂赌博机}
\begin{itemize}
    \item 只有一个状态,K个动作
    \item 每个摇臂的奖赏服从某个期望未知的分布
    \item 执行有限次动作
    \item 最大化累积奖赏
\end{itemize}

强化学习面临的主要困难:探索-利用窘境 (Exploration-Exploitation dilemma)
\begin{itemize}
    \item 探索:估计不同摇臂的优劣 (奖赏期望的大小)
    \item 利用:选择当前最优的摇臂
\end{itemize}

在探索与利用之间进行折中:
\begin{itemize}
    \item $\epsilon$-贪心
    \item Softmax
\end{itemize}

\subsubsection{\texorpdfstring{$\epsilon$}. 贪心}
\begin{itemize}
    \item 以$\epsilon$的概率探索:均匀随机选择一个摇臂
    \item 以$1-\epsilon$的概率利用:选择当前平均奖赏最高的摇臂
\end{itemize}

\subsubsection{Softmax}
基于当前已知的摇臂平均奖赏来对探索与利用折中, 若某个摇臂当前的平均奖赏越大,则它被选择的概率越高. 

\subsubsection{离散空间强化学习}
离散空间状态空间、离散动作空间上的多步强化学习任务

方法:
\begin{itemize}
    \item 每个状态上动作的选择看作一个 K-摇臂赌博机问题
    \item K-摇臂赌博机算法奖赏函数:强化学习任务的累积奖赏
\end{itemize}
局限:未考虑马尔科夫决策过程

\subsection{有模型学习}
有模型学习 (model-based learning)
\begin{align*}
    E=\braket{X, A, P, R}
\end{align*}
$X, A, P, R$ 均已知. 

强化学习的目标:找到使累积奖赏最大的策略$\pi$

策略评估:使用某策略所带来的累积奖赏
\begin{itemize}
    \item 状态值函数:从状态$x$出发,使用策略$\pi$所带来的累积奖赏
    \item 状态-动作值函数:从状态$x$出发,执行动作$a$后再使用策略
    $\pi$所带来的累积奖赏
\end{itemize}

策略改进:将非最优策略改进为最优策略

策略迭代 (policy iteration): 求解最优策略的方法
\begin{enumerate}
    \item 随机策略作为初始策略
    \item 策略评估+策略改进+策略评估+策略改进+...
    \item 直到策略收敛
\end{enumerate}

\subsection{免模型学习}
免模型学习 (model-free learning):
\begin{itemize}
    \item 转移概率,奖赏函数未知
    \item 甚至环境中的状态数目也未知
    \item 假定状态空间有限
\end{itemize}

免模型学习所面临的困难: 策略无法评估, 无法通过值函数计算状态-动作值函数, 机器只能从一个起始状态开始探索环境. 

解决困难的办法: 多次采样, 直接估计每一对状态-动作的值函数, 在探索过程中逐渐发现各个状态


两种著名的免模型学习方法:
\begin{itemize}
    \item 蒙特卡罗强化学习
    \item 时序差分学习
\end{itemize}

\subsubsection{蒙特卡罗强化学习}
采样轨迹,用样本均值近似期望

策略评估:蒙特卡罗法
\begin{enumerate}
    \item 从某状态出发,执行某策略
    \item 对轨迹中出现的每对状态-动作,记录其后的奖赏之和
    \item 采样多条轨迹,每个状态-动作对的累积奖赏取平均
\end{enumerate}

策略改进:换入当前最优动作

蒙特卡罗强化学习可能遇到的问题:轨迹的单一性. 解决问题的办法: $\epsilon$-贪心
\begin{itemize}
    \item 同策略(on-policy):被评估与被改进的是同一个策略
    \item 异策略(off-policy):学习过程是基于行为策略产生的数据,但是学习的目标是优化目标策略 (用重要性采样技术)
\end{itemize}
\subsubsection{时序差分学习}
增量式地进行状态-动作值函数更新, 也是$\epsilon$-贪心
\begin{itemize}
    \item 同策略:Sarsa算法
    \item 异策略:Q-学习 (Q-learning)
\end{itemize}

\paragraph{Sarsa算法}需要前一步的状态(state) 、前一步的动 (action)、奖赏值(reward) 、当前状态 (state) 、将要执行的动作(action)

\paragraph{Q-学习算法}该算法评估的是$\epsilon$-贪心策略,而执行的是原始策略

\subsection{值函数近似}
若状态空间是连续(无限)的, 需要使用值函数近似. 但为简便假定:
\begin{itemize}
    \item 状态空间 $X=\R^n$
    \item 考虑线性近似
    \item 行为空间有限
\end{itemize}

值函数近似: 将值函数表达为状态的线性函数
\begin{align*}
    V_{\bm \theta}(\bx)=\bm{\theta}^\top \bx
\end{align*}
$\bm{\theta}$ 为参数向量, $\bx$ 为状态向量. 

用最小二乘误差来度量学到的值函数与真实的值函数$V^\pi$之间的近似程度
\begin{align*}
    E_{\bm\theta}=\mathbb{E}_{\bx\sim\pi}\left[ \left( V^\pi(\bx)-V_{\bm\theta}(\bx) \right)^2 \right]
\end{align*}

用梯度下降法更新参数向量,求解优化问题.

单个样本更新策略
\begin{align*}
    \bm\theta=\bm\theta+\alpha(V^\pi(\bx)-V_{\bm\theta}(\bx) )\bx
\end{align*}

借助时序差分学习,使用估计的值函数 $V^\pi(\bx)=r+\gamma V^\pi(\bx')$ 代替真实值函数

\begin{align*}
    \bm\theta&=\bm\theta+\alpha(r+\gamma V^\pi(\bx)-V_{\bm\theta}(\bx) )\bx\\
    &=\bm\theta+\alpha(r+\gamma\bm{\theta}^\top\bx'-\bm{\theta}^\top\bx )\bx\\
\end{align*}

线性值函数近似Sarsa算法. 可以通过引入核方法实现非线性值函数近似。

\subsection{模仿学习}
强化学习任务中多步决策的搜索空间巨大,基于累积奖赏来学习很多步之前的合适决策非常困难. 直接模仿人类专家的状态-动作对来学习策略. 

\begin{enumerate}
    \item 利用专家的决策轨迹,构造数据集D:状态作为特征,动作作为标记
    \item 利用数据集D,使用分类/回归算法即可学得策略
    \item 将学得的策略作为初始策略
    \item 策略改进,从而获得更好的策略
\end{enumerate}

强化学习任务中,设计合理的符合应用场景的奖赏函数往往相当困难. 从人类专家提供的范例数据中反推出奖赏函数. i.e. 逆强化学习 (inverse reinforcement learning)
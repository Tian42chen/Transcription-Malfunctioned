\newpage
\section{概率图模型}
概率模型:提供了一种描述框架,将学习任务归结于计算变量的概率分布

推断:在概率模型中,利用已知变量推测未知变量的分布


\subsection{隐马尔可夫模型}
概率图模型:
一类用图来表达变量相关关系的概率模型。它以图为表示工具,
最常见的是用一个结点表示一个或一组随机变量,结点之间的边
表示变量间的概率相关关系,即“变量关系图”。
\begin{itemize}
    \item 使用有向无环图表示变量间的依赖关系,称为有向圈模型或
    贝叶斯网
    \item 使用无向国表示变量间的相关关系,称为无向图模型或马尔
    可夫网
\end{itemize}

对于 隐马尔可夫模型
\begin{itemize}
    \item 状态变量: 假定状态变量是隐藏的、不可被观测的,因此状态变量亦称隐变量。
    \item 观测变量: 可以是离散型也可以是连续型。
\end{itemize}
马尔可夫链:(\textit{现在决定未来})
系统下一时刻的状态仅由当前状态决定,不依赖于以往的任何状态. 基于这种依赖关系,所有变量的联合概率分布为
\begin{align*}
    P(x_1, y_1,\dots,x_n,y_n)=P(y_1)P(x_1|y_1)\prod_{i=2}^nP(y_i|y_{i-1}P(x_i|y_i))
\end{align*}


欲确定一个隐马尔可夫模型还需以下三组参数:
\begin{itemize}
    \item 状态转移概率: 在各个状态间转换的概率
    \item 输出现测概率: 根据当前状态获得各个观测值的概率
    \item 初始状态概率
\end{itemize}

观测序列产生:
\begin{enumerate}\small
    \item 设置$t=1$ , 并根据初始状态概率$\bm\pi$选择初始状态$y_1$;
    \item 根据状态$y_t$和输出观测概率$\bm B$ 选择观测变量取值$x_t$;
    \item 根据状态$y_t$和状态转移矩阵$\bm A$ 转移模型状态,即 确 定 $y_{t+1}$;
    \item 若$t<n$ 设置$t=t+1$ , 并 转到第 2) 步,否 则 停止. 
\end{enumerate}

\subsection{马尔可夫随机场}
马尔可夫随机场(Markov Random Field,简称MRF):
典型的马尔可夫网,这是一种著名的无向图模型。图中每个结点
表示一个或一组变量,结点之间的边表示两个变量之间的依赖关
系。马尔可夫随机场有二组势函数,亦称“因子”,这是定义在变
量子集上的非负实函数,主要用于定义概率分布函数。

可得到两个推论:
\begin{itemize}
    \item 局部马尔可夫性:给定某变量的邻接变量,则该变量条件独立
    于其他变量
    \item 成对马尔可夫性:给定所有其他变量,两个非邻接变量条件独
    立
\end{itemize}


\subsection{条件随机场}
条 件 随 机 场 (Conditional Random Field ,简 称 CRF )是一种判别式无向图
模 型, 试图对多个变量在给定观测值后的条件概率进行建模

\subsection{学习与推断}
边际化:给定参数$\Theta$求解某个变量$\bx$的分布,就变成对联合分布中其他无关变量进行积分的过程. 推断问题的目标就是计算边际概率或条件概率

两种代表性的精确推断方法:
\begin{itemize}
    \item 变量消去
    \item 信念传播
\end{itemize}

\subsection{近似推断}
\begin{itemize}
    \item 采样(sampling) ,通过使用随机化方法完成近似
    \item 使用确定性近似完成近似推断,典型代表为变分推断
    (variational inference)
\end{itemize}
\subsubsection{MCMC采样}
概率图模型中最常用的采样技术是马尔可夫链蒙特卡罗(Markov Chain Monte Carlo ,简 称 MCMC)方法. MCMC方法的关键就在于通过构造“平稳分布为$p$ 的马尔可夫链”来产生样本. 

Metropolis-Hastings (简称MH )算 法 是 MCMC的 重 要 代 表. .它 基 于 “拒
绝 采 样 ”(reject sampling)来 逼 近 平 稳 分 布 $p$. 吉布斯采样(Gibbs sampling)有 时 被 视 为 M H 算法的特例. 


\subsubsection{变分推断}
变分推断通过使用已知简单分布来逼近需推断的复杂分布,并通过限制近
似分布的类型,从而得到一种局部最优、但具有确定解的近似后验分布


\subsection{话题模型}
话题模型(topic model)是一族生成式有向图模型,主要用于处理离散型的
数据(如文本集合),在信息检索、 自然语言处理等领域有广泛应用. 隐狄利克
雷分配模型(Latent Dirichlet Allocation,简 称 LDA)是话题模型的典型代表.
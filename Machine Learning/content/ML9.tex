\newpage
\section{聚类}
\subsection{聚类任务}
目标:将数据样本划分为若干个通常不相交的“簇”(cluster)

\subsection{性能度量}
聚类性能度量,亦称聚类“有效性指标”(validity index)
\begin{itemize}
    \item 外部指标 (external index)
    \subitem 将聚类结果与某个“参考模型”(reference model) 进行比较
    \item 内部指标 (internal index)
    \subitem 直接考察聚类结果,无参考模型
\end{itemize}



\subsection{距离计算}
聚类来自于分组,分组来自于合理度量,度量来自于距离, 因此距离对聚类很本质. 

距离度量 (distance metric) 需满足的基本性质:
\begin{itemize}
    \item 非负性: $dist(\bx_i,\bx_j)\ge0$
    \item 同一性: $dist(\bx_i,\bx_j)=0 \iff \bx_i=\by_i$
    \item 对称性: $dist(\bx_i,\bx_j)=dist(\bx_j,\bx_i)$
    \item 直递性: $dist(\bx_i,\bx_j)\le dist(\bx_i,\bx_k)+dist(\bx_k,\bx_j)$
\end{itemize}

常用距离形式:
\begin{itemize}
    \item 闵可夫斯基距离 (Minkowski distance)
    \begin{align*}
        dist_{mk}(\bx_i,\bx_j)=\left( \sum_{u=1}^n|x_{iu}-x_{ju}|^p \right)^{\frac{1}{p}}
    \end{align*}
    \subitem $p = 2$: 欧氏距离(Euclidean distance)
    \subitem $p = 1$: 曼哈顿距离(Manhattan distance)
    \item 对无序(non-ordinal)属性,可使用 VDM (Value Difference Metric)
    \subitem 令 $m_{u,a}$ 表示属性 $u$ 上取值为 $a$ 的样本数,$m_{u,a,i}$ 表示在第 $i$ 个样本簇中在属性 $u$ 上取值为 $a$ 的样本数,$k$ 为样本簇数,则属性 $u$ 上两个离散值 $a$ 与 $b$ 之间的 VDM 距离为
    \begin{align*}
        VMD_p(a,b)=\sum_{i=1}^k\left| \frac{m_{u,a,i}}{m_{u,a}}-\frac{m_{u,b, i}}{m_{u,b}} \right|^p
    \end{align*}
    \item 对混合属性,可使用 MinkovDM
    \begin{align*}
        MinkovDM_p(\bx_i,\bx_j)=&\left( \sum_{u=1}^{n_c}|x_{iu}-x_{ju}|^p\right. \\
        &\left.+ \sum_{u=n_c+1}^n VDM_p(x_{iu}, x_{ju}) \right)^{\frac{1}{p}}
    \end{align*}
\end{itemize}


\subsection{原型聚类}
原型=簇中心,有簇中心的聚类方法

\subsubsection{k-均值聚类 (k-means)}
每个簇中心以该簇中所有样本点的“均值”表示. 

步骤:
\begin{enumerate}
    \item 随机选取k个样本点作为簇中心
    \item 将其他样本点根据其与簇中心的距离,划分给最近的簇
    \item 更新各簇的均值向量,将其作为新的簇中心
    \item 若所有簇中心未发生改变,则停止;否则执行 2)
\end{enumerate}

\subsubsection{学习向量量化 (LVQ)}
Learning Vector Quantization. 也是试图找到一组原型向量来刻画聚类结构,但数据样本带有类别标记. 通过聚类来形成类别的“子类”结构;每个聚类对应于类别的一个子类. 

\subsubsection{高斯混合聚类 (GMM)}
Gaussian Mixture Clustering. 采用高斯概率分布来表达聚类原型,簇中心=均值,簇半径=方差. 参数估计可采用极大似然法与EM算法. 

\subsection{密度聚类}
划成多个等价类,未必有簇中心

\subsubsection{DBSCAN}
通过邻域建立样本的可达路径,形成等价类(连通分支). 

DBSCAN将 “簇 ”定义为:由密度可达关系导出的最大的密度相连样本集合

\subsection{层次聚类}
聚类效果跟抽象的粒度有关,形成多层次聚类

\subsubsection{AGNES (AGglomerative NESting)}
从最细的粒度开始(一个样本一个簇),逐渐合并相似的簇,直到最粗的粒度(所有样本一个簇)

步骤:
\begin{enumerate}
    \item 将每个样本点作为一个簇
    \item 合并最近的两个簇
    \item 若所有样本点都存在与一个簇中, 则停止; 否则 2)
\end{enumerate}
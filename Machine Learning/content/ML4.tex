\newpage
\section{决策树}
\subsection{决策树简介(基本流程)}
处理未见的能力

递归构建决策树, 三个条件停止递归:
\begin{enumerate}
    \item 
\end{enumerate}




\subsection{决策树算法的关键:划分选择}
最优划分属性

希望 purity 愈来愈高. 

\subsubsection{信息增益}%TODO 练习计算

偏向于类目多的属性

\subsubsection{增益率}

偏向于类目少的属性, 相当于给增益做了规范化

\subsubsection{基尼系数}
反映了从数据集中随机抽取两个样本, 其类别标记不一致的概率; 基尼指数越小, 说明越纯

\subsection{克服过拟合的问题:剪枝处理}
没有先验, 非常易过拟合. 

\subsubsection{预剪枝}
边建树,边剪枝

\subsubsection{后剪枝}
先建树,后剪枝

\subsection{处理多种类型数据:连续与缺失值}
离散化

\subsubsection{二分化}

\subsubsection{缺失值}
两个问题:
\begin{enumerate}\small
    \item 如何在属性值缺失的情况下进行划分属性选择? 跟传统决策树一致,只不过仅在有属性值的子集上计算信息增益,不考虑无属性值的样本
    \item 给定划分属性,若样本在该属性上的值缺失,如何对样本进行划分?就是让同一个样本以不同的概率划入到不同的子节点中去
\end{enumerate}

\subsection{决策树的变体:多变量决策树}
非叶节点不再是仅对某个属性,而是对属性的线性组合. 

多变量本质就是模型, 相当于叶节点是小模型. 
\newpage
\section{活动星系核}
\begin{itemize}
    \item 普通星系
    \item 活动星系
\end{itemize}

活动星系的典型特征
\begin{enumerate}
    \item 
\end{enumerate}
%TODO P4

主要分类
\begin{enumerate}
    \item 
\end{enumerate}
%TODO P5

\subsection{活动星系与普通星系的比较}

\subsubsection{密度}
%TODO P6
\subsubsection{高光度}
%TODO P7
\subsubsection{快速光变}
%TODO P8
\subsubsection{非热连续辐射, 高偏正辐射 }
%TODO P9-10
说明辐射并不是来自恒星

同步加速辐射

\subsubsection{射电波段特殊形态}
%TODO P11

\subsubsection{强发射线}
%TODO P12

\subsection{活动星系核的主要类型}

\subsubsection{赛弗特星系}
%TODO P14

\paragraph{与正常星系发射线的比较}
%TODO P15

\paragraph{赛弗特星系分类}
%TODO P16

\subsubsection{射电星系}
%TODO P17

\paragraph{形态特征}
%TODO P18

\paragraph{基本特征}
%TODO P19

e.g. Cygnus, M87, Centaurus
%TODO P20-22

\subsection{类星体}
%TODO P23-26

\subsubsection{寄主星系}
%TODO P28

\subsubsection{光谱 - Gunn-Peterson 效应}
%TODO P29-30

\subsection{活动星系核理论模型}

\subsubsection{AGN 的能源问题}
%TODO P31-33

\paragraph{爱丁顿光度}
%TODO P34

\paragraph{施瓦西半径}
%TODO P35

\paragraph{爱丁顿吸积}
%TODO P35

\paragraph{黑洞引力能释放}
%TODO P36

\paragraph{宇宙的终极能源 - 引力能}
%TODO P38

\subsubsection{活动星系统一模型}
%TODO P39
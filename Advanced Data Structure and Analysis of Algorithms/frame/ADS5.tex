\newpage
\section{Binomial Queue}
A binomial queue is not a heap-ordered tree, but rather a collection of heap-ordered trees, known as a forest. Each heap-ordered tree is a \hl{binomial tree}. 

A binomial tree of height 0 is a one-node tree. A binomial tree, $B_k$, of height $k$ is formed by attaching a binomial tree, $B_{k-1}$, to the root of another binomial tree, $B_{k-1}'$. 

$B_k$ consists of a root with $k$ children, which are $B_0$, $\dots$, $B_{k-1}$. $B_k$ has exactly $2^k$ nodes. The number of nodes at depth $d$ is $\begin{pmatrix}
    k\\d
\end{pmatrix}$ (Binomial coefficient).

A priority queue of any size can be uniquely represented by a collection of binomial trees. (二进制)

\subsection{Operations}

\subsubsection{FindMin}
The minimun key is in one of the roots. There are at most $\left\lceil \log N \right\rceil$ roots, hence $T_p=O(\log N)$. Or just remember the minimun, it's $O(1)$. 

\subsubsection{Merge}
二进制加法. 合并是任意的. $T_p=O(\log N)$. 
Must keep the trees in the binomial queue sorted by height.

\subsubsection{Insert}
If the smallest nonexistent binomial tree is $B_i$ , then $T_p=Const(i+1)$. Performing $N$ Inserts on an initially empty binomial queue will take $O(N)$ worst-case time.  Hence the average time is constant.

\subsubsection{DeleteMin}
\begin{enumerate}
    \item FindMin in $B_k$
    \item Remove $B_k$ from H gets $H'$
    \item Remove root from $B_k$ gets $H''$
    \item Merge ($H'$,$H''$)
\end{enumerate}

\subsection{Implementation}
\begin{lstlisting}[language={c++},title={Binomial Queue}]
typedef struct BinNode *Position;
typedef struct Collection *BinQueue;
typedef struct BinNode *BinTree;

struct BinNode{
    ElementType Element;
    Position LeftChild;
    Position NextSibling;
};

struct Collection{
    int CurrentSize;/* total number of nodes */
    BinTree TheTrees[MaxTrees];
};
\end{lstlisting}

\begin{lstlisting}[language={c++},title={MergeTrees}]
BinTree MergeTrees(BinTree T1, BinTree T2){
    if(T1->Element>T2->Element) return MergeTrees(T2, T1);
    T2->NextSibling=T1->LeftChild;
    T1->LeftChild=T2;
    return T1;
}
\end{lstlisting}

\begin{lstlisting}[language={c++},title={Merge}]
BinQueue Merge(BinQueue H1, BinQueue H2){

}
\end{lstlisting}

\begin{lstlisting}[language={c++},title={DeleteMin}]
BinQueue DeleteMin(BinQueue H){

}
\end{lstlisting}
\subsection{Analysis}
\begin{claim}
A binomial queue of $N$ elements can be built by $N$ successive insertions in $O(N)$ time. 
\end{claim}

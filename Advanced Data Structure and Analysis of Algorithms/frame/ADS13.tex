\newpage
\section{Randomized Algorithms}
\begin{enumerate}\small
    \item What to Randomize?
    \subitem Average-case Analysis: The world behaves randomly --- randomly generated input solved by traditional algorithm
    \subitem Randomized Algorithms: The algorithm behaves randomly --- make random decisions as the algorithm processes the worst-case input
    \item Why Randomize?
    \subitem Efficient deterministic algorithms that always yield the correct answer are a special case of 
    \begin{itemize}
        \item efficient randomized algorithms that only need to yield the correct answer with high probability. \item randomized algorithms that are always correct, and run efficiently in expectation. 
    \end{itemize}
\end{enumerate}

\subsection{A Quick Review}
\begin{itemize}
    \item $Pr[A]:=$ the probability of the event $A$
    \item $\bar{A}:=$ the complementary of the event $A$ ($A$ did not occur )
    \begin{align*}
        Pr[A]+Pr[\bar{A}]=1
    \end{align*}
    \item $E[X]:=$ the expectation (the ``average value'') of the random variable X
    \begin{align*}
        E[X]=\sum_{j=0}^\infty j \cdot Pr[X=j]
    \end{align*}
\end{itemize}

\subsection{The Hiring Problem}
\begin{itemize}
    \item Hire an office assistant from headhunter 
    \item Interview a different applicant per day for $N$ days
    \item Interviewing Cost = $C_i$  $\ll$  Hiring Cost = $C_h$
    \item Analyze interview \& hiring cost 
\end{itemize}
Assume M people are hired.

Total Cost: $O(NC_i + MC_h)$

\subsubsection{Naïve Solution}
\begin{lstlisting}[language={c++}]
int Hiring(EventType C[], int N){ /* candidate 0 is a least-qualified dummy candidate */  
    int Best = 0;
    int BestQ = the quality of candidate 0;
    for(i=1;i<=N;i++){
        Qi = interview(i); /* Ci */
        if(Qi>BestQ){
            BestQ = Qi;
            Best = i;
            hire(i);  /* Ch */
        }
    }
    return Best;
}
\end{lstlisting}
Worst case: The candidates come in increasing quality order $O(NC_h)$

\subsubsection{Assume candidates arrive in random order}
Randomness assumption: any of first $i$ candidates is equally likely to be best-qualified so far. 
\begin{align*}
    X &= \text{number of hires}\\
    E[X]&=\sum_{j=1}^N j\cdot Pr[X=j]\\
    X_i&=\left\{ \begin{array}[c]{ll}
        1 & \text{if candidate $i$ is hired}\\
        0 & \text{if candidate $i$ isn't hired}
    \end{array} \right.\\
    \Rightarrow&X=\sum_{i=1}^N X_i \\
    & E[X_i]=Pr[\text{candidate $i$ is hired}]=\frac{1}{i}\\
    \Rightarrow&E[X]=E\left[ \sum_{i=1}^NX_i \right]=\sum_{i=1}^N E[X_i]\\
    \Rightarrow& O(C_h \ln N + N C_i)
\end{align*}
Radomized Algorithm: takes time to randomly permute the list of candidates;

\paragraph{Radomized Permutation Algorithm}\quad 

\textbf{Target}: Permute array $A[]$. 

Assign each element $A[ i ]$ a random priority $P[ i ]$, and sort. 
\begin{lstlisting}[language={c++}]
void PermuteBySorting(ElemType A[], int N){
    for(i=1;i<=N;i++)
        A[i].P = 1 + rand()%(`$N^3$`); 
        /* makes it more likely that all priorities are unique */
    Sort A; /* using P as the sort keys*/
}
\end{lstlisting}
\begin{claim}
    PermuteBySorting produces a uniform random permutation of the input, assuming all priorities are distinct.
\end{claim}

\subsubsection{Online Hiring Algorithm – hire only once}
\begin{lstlisting}[language={c++}]
int OnlineHiring(EventType C[], int N, int k ){
    int Best = N;
    int BestQ = -`$\infty$` ;
    for(i=1;i<=k;i++){
        Qi = interview(i);
        if(Qi>BestQ)BestQ = Qi;
    }
    for(i=k+1;i<=N;i++){
        Qi = interview(i);
        if(Qi>BestQ){
            Best = i;
            break;
        }
    }
    return Best;
}
\end{lstlisting}

$S_i:=$ the $i$-th applicant is the best. 

$S_i$ is true if
\begin{align*}
    &\{ A:= \text{the best one is at position $i$} \} \\
    \cap &\{ B:=  \text{no one at positions $k+1 \sim i-1$ are hired} \}    
\end{align*}
$A$ and $B$ are independent. 

\begin{align*}
    Pr[S_i]&=Pr[A\cap B]=Pr[A]\cdot Pr[B]\\
    &=\frac{1}{N}\frac{k}{(i-1)}=\frac{k}{N(i-1)}\\
    Pr[S]&=\sum_{i=k+1}^N Pr[S_i]=\sum_{i=k+1}^N\frac{k}{N(i-1)}=\frac{k}{N}\sum_{i=k}^{N-1}\frac{1}{i}
\end{align*}

The probability we hire the best qualified candidate for a given $k$:
\begin{align*}
    \frac{k}{N}\ln\left( \frac{N}{k} \right)\le Pr[S] \le \frac{k}{N}\ln\left( \frac{N-1}{k-1} \right)
\end{align*}

The best value of $k$ to maximize the above probability:
\begin{align*}
    f(k)&=\frac{k}{N}\ln\left( \frac{N}{k} \right)\\
    f'(k)&=\frac{1}{N}\ln\left( \frac{N}{k} \right)-\frac{1}{N}=0\\
    k&=\frac{N}{e}\\
    f_{max}\left(\frac{N}{e}\right)&=\frac{1}{e}
\end{align*}

\subsection{Quicksort}
Deterministic Quicksort
\begin{itemize}
    \item $\Theta (N^2)$ worst-case running time
    \item $\Theta(N \log N)$ average case running time, assuming every input permutation is equally likely    
\end{itemize}

Central splitter $:=$ the pivot that divides the set so that each side contains at least $n/4$

Modified Quicksort $:=$ always select a central splitter before recursions

\begin{claim}
    The expected number of iterations needed until we find a central splitter is at most 2.
\end{claim} 

\begin{align*}
    Pr[\text{find a central splitter}]=\frac{1}{2}
\end{align*}

Type $j$ : the subproblem $S$ is of type $j$ if
\begin{align*}
    N\left( \frac{3}{4} \right)^{j+1}\le |S|\le N\left( \frac{3}{4} \right)^j
\end{align*}


\begin{claim}
    There are at most $\left( \frac{4}{3} \right)^{j+1}$ subproblems of type $j$.
\begin{align*}
    &\left.\begin{array}[c]{l}
        E[T_{type\ j}]=O\left( N\left( \frac{3}{4} \right)^j \right)\times \left( \frac{4}{3} \right)^{j+1}=O(N)\\
        \text{Number of different types }=\log_{4/3}N=O(\log N)
    \end{array}\right\}\\
    &=O(N\log N)
\end{align*}
\end{claim} 

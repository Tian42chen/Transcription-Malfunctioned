\newpage
\section{Network Security}
Basic knowledge
\subsection{Cryptography}
\subsubsection{Kerckhoff's principle}
All algorithms must be public; only the keys are secret.

\subsubsection{Substitution ciphers}
In a substitution cipher, each letter or group of letters is replaced by another letter or group of letters to disguise it.

The basic attack takes advantage of the statistical properties of natural languages.

\subsubsection{Transposition ciphers}
Transposition ciphers, in contrast, reorder the letters but do not disguise them. The cipher is keyed by a word or phrase not containing any repeated letters.

To break a transposition cipher, the cryptanalyst must 
\begin{enumerate}
    \item first be aware that he is dealing with a transposition cipher.
    \item The next step is to make the guess at the number of columns (the key length).
    \item The remaining step is to order the columns.
\end{enumerate}


\subsubsection{One-time pads}
First choose a random bit string as the key. Then convert the plaintext
into a bit string, for example, by using its ASCII representation.
Finally, compute the XOR of these two strings, bit by bit.

The reason derives from information theory: there is simply no
information in the message because all possible plaintexts of the given
length are equally likely.

% \subsubsection{Quantum cryptograph}
\subsubsection{Two fundamental cryptographic principles}
\begin{enumerate}
    \item All the encrypted messages must contain some redundancy.
    \item Measures must be taken to ensure that each message received can be verified as being fresh.
\end{enumerate}

\subsection{Symmetric-key algorithms}
In symmetric-key algorithms, they use the same key for encryption and decryption.


\subsubsection{DES}
要经历 19 个阶段. 

\subsubsection{Triple DES}
whitening, 不确定性达到最大值. 

\subsubsection{Rijndael}

很重要, 但不讲了.jpg

\subsection{Public-key algorithms}

These requirements can be stated simply as follows:
\begin{enumerate}
    \item $D(E(P))=P$
    \item It is exceedingly difficult to deduce $D$ from $E$.
    \item $E$ cannot be broken by a chosen plaintext attack.
\end{enumerate}

Public-key cryptography requires each user to have two keys: a public key, used by the entire world for encrypting message to be sent to that user, and a private key, which the user needs for decrypting messages.

\subsubsection{RSA}%TODO 33-42

\subsection{Digital Signatures}
The authenticity of many legal, financial, and other
documents is determined by the presence or absence of an
authorized handwritten signature.

The basic requirements of digital signatures:
\begin{enumerate}
    \item The receiver can verify the claimed identity of the sender.
    \item The sender cannot later repudiate the contents of the message.
    \item The receiver cannot possibly have concocted the message
    himself.
\end{enumerate}

\subsubsection{Symmetric-Key Signatures}


Replay Attack


\subsubsection{Public-Key Signatures}


\subsubsection{Message Digests}
One criticism of signature methods is that they often couple two
distinct functions: authentication and secrecy. 


SHA-1 and SHA-2


\subsubsection{The Birthday Attack}
The birthday attack is that with two different plaintexts, but
have the same message digests.


\subsection{Management of public keys}
\subsubsection{Certificates}

\subsubsection{Public Key Infrastructures}

\newpage
\section{Quantum Error Correction}

\subsection{Error Representation and Correction}
Errors for a single qubit can be represented by unexpected $Z, X$, and $Y$ gates (phase error, bit-flip error, and combined bit-flip and phase error, respectively) in a quantum circuit.

Redundancy in coding can reduce the error rate from $p\ll 1$ to $O(p^2)$. The key in quantum error correction lies in how to replicate and extract the information.

\subsubsection{Representation of Errors}
Interaction with the environment will transform and entangle the states of the qubit and its environment. In general, we expect the following process:
\begin{align*}
    \ket{e_0}\ket{0}&\rightarrow \ket{e_1}\ket{0}+\ket{e_2}\ket{1}\\
    \ket{e_0}\ket{1}&\rightarrow \ket{e_3}\ket{0}+\ket{e_4}\ket{1}
\end{align*}
where $\ket{e_{0-4}}$ are the states of the environment. 

For practical quantum computing, we expect weak coupling between the qubit and its environment. Hence we expect
\begin{align*}
    \sqrt{\left|\braket{e_0|e_1}\right|^2}\approx \sqrt{\left|\braket{e_0|e_4}\right|^2}\approx 1\\
    \braket{e_2|e_2}, \braket{e_3|e_3}\ll 1
\end{align*}
To discuss the interaction with the environment, it is useful to rewrite projection operators, which satisfy
\begin{align*}
    P_0\ket{\psi}=\ket{0}\braket{0|\psi},\ P_1\ket{\psi}=\ket{1}\braket{1|\psi}
\end{align*}
as
\begin{align*}
    P_\alpha=\frac{I+(-1)^\alpha Z}{2}
\end{align*}

We can now write
\begin{align*}
    \ket{e_0}\ket{\psi}\rightarrow\left[ \ket{e_1}I+\ket{e_2}X \right]P_0\ket{\psi}+\left[ \ket{e_3}X+\ket{e_4}I \right]P_1\ket{\psi}
\end{align*}
where $\ket{e}O\ket{\psi}\equiv \ket{e}\otimes O\ket{\psi}$. 

Plugging in the explicit form of $P_\alpha$, we find
\begin{align*}
    \ket{e_0}\ket{\psi}\rightarrow& \left[ \frac{\ket{e_1}+\ket{e_4}}{2}I+\frac{\ket{e_1}-\ket{e_4}}{2}Z\right. \\
    &\left. +\frac{\ket{e_3}+\ket{e_2}}{2}X+\frac{\ket{e_3}-\ket{e_2}}{2}Y \right] \ket{\psi}\\
    \equiv & \left( \ket{d}I+\ket{a}X+\ket{b}Y+\ket{c}Z \right)\ket{\psi}
\end{align*}
The four operators $I, Z, X$ and $Y$ correspond to no error, phase error, bit-flip error, and combined bit-flip and phase error, respectively.

We notice that
\begin{align*}
    \ket{d}=\frac{\ket{e_1}+\ket{e_4}}{2}\approx\ket{e_0}
\end{align*}
but $\ket{a}$, $\ket{b}$, and $\ket{c}$, which are associated with various errors, are ket-vectors with small amplitudes. This is consistent with our weak-coupling assumption. For one qubit, there are 3 terms of errors.

When we generalize the result to 1-bit errors for n-bit codewords $\ket{\Psi_n}$, we can write
\begin{align*}
    \ket{e_0}\ket{\Psi_n}\rightarrow & \ket{d}\ket{\Psi_n}+\\
    &\sum_{i=1}^n(\ket{a_i} X_i \ket{\Psi_n} + \ket{b_i} Y_i \ket{\Psi_n} + \ket{c_i} Z_i \ket{\Psi_n} )
\end{align*}
Altogether, there are $(3n + 1)$ terms.

\subsubsection{Error Correction}
A classical code is defined as being a set of bit strings with certain properties.

Before 1995, it was believed that the concept of error correction could not apply to quantum systems, for the following reasons:
\begin{enumerate}\small
    \item Quantum states collapse when measured.
    \item Errors are continuous.
    \item Quantum states cannot be cloned.
\end{enumerate}

It turns out there are ways around all of these objections.
\begin{enumerate}\small
    \item The part of the Hilbert space containing the quantum information to be preserved need not be measured; only the effect of the environment need be determined by a measurement. 
    \item Using entangled states allow errors to be made orthogonal and distinguishable. 
    \item Entanglement also replaces the role played by redundant copies in classical error correction.
\end{enumerate}

To encode each logic qubit in multiple physical qubit, we replace $\ket{0}$ and $\ket{1}$ by the codewords, e.g.
\begin{align*}
    \ket{\tilde{0}}=\ket{0}\ket{0}\ket{0}=\ket{000}\\
    \ket{\tilde{1}}=\ket{1}\ket{1}\ket{1}=\ket{111}
\end{align*}
One (or two) bit flip moves the states to the non-encoding space $\ket{001}, \ket{010}, \ket{100}, \ket{110}, \ket{101}, \ket{011}$, therefore can be detected. But as measurement can alter a state, one must come up with less obvious form of measurement.

In addition to bit-flip errors, there can also be phase errors,
e.g.
\begin{align*}
    \frac{\ket{0}+\ket{1}}{\sqrt{2}}\rightarrow\frac{\ket{0}-\ket{1}}{\sqrt{2}}
\end{align*}

A generalization of the 3-bit coding to 9-bit can correct both bit-flip and phase errors. See, e.g., D. Gottesman, arXiv:0904.2557, Sec. 2.1.

\subsection{The 3-Bit Repetition Code}
While measurement can alter a state, there exist less obvious forms of measurement for error detection. In a repetition code, one can use CNOT gates to extract the information without directly measuring the qubits.

Errors, in general, leads to information leakage into the non-encoding space. One can correct the errors by returning to the encoding space without revealing any information of the logical qubit.

Preparation
\begin{align*}
    \ket{\psi}\equiv \alpha\ket{000}+\beta\ket{111}
\end{align*}
%TODO P14

Occasionally, a bit is flipped (with a small probability $p$).
\begin{align*}
    \ket{\psi_1}=X_1\ket{\psi}=\alpha\ket{100}+\beta\ket{011}\\
    \ket{\psi_2}=X_2\ket{\psi}=\alpha\ket{010}+\beta\ket{101}\\
    \ket{\psi_3}=X_3\ket{\psi}=\alpha\ket{001}+\beta\ket{110}
\end{align*}

Detection: We cannot measure the three qubits, but can use CNOT gates to extract the information.
%TODO P15-19

In addition to detection, we can directly correct the error by the following circuit without measuring.
%TODO P20

The essence is that we can restore $\ket{\psi}$ without revealing any information about $\alpha$ and $\beta$.


\subsection{Toward Fault-Tolerant Quantum Computation}
Provided the noise in individual quantum gates is below a certain constant threshold, it is possible to reliably perform an arbitrary quantum computation.

In the modern surface code, errors are allowed to exist and propagate (hence fault-tolerant), as long as they are sparse. Once identified, classical control software can be used to track and correct the errors separately.

\subsubsection{Fault-Tolerant Quantum Computation}
Even though the repetition code cannot correct all possible errors, it convinces us that noise is not a serious problem in principle; error correction can reduce the error probability from $p$ to $O(p^2)$. 

If we further introduce hierarchical procedures, the error probability can be ultimately reduced to as low a level as desired, as long as the original error rate $p$ is below some threshold $p_{th}$ (with physically reasonable assumptions).

This is known as the threshold theorem, as was also important for early-day classical computation.

\subsubsection{Understanding Error Detection}
To understand the connection between error detection and operator measurement, we digress to the following circuit for measurement of an operator $U$ acting on a single qubit.
%TODO P24

Here, we suppose $U$ has eigenvalues $\pm 1$ and corresponding eigenvector $\ket{u_\pm}$. So a single-qubit state $\ket{\psi}$ can be generically expressed as $\alpha\ket{u_+}+\beta\ket{u_-}$. 

Following the circuit from left to right, the two-qubit state evolves as
%TODO P25
\begin{align*}
    1
\end{align*}
When the measurement result is 0, the circuit outputs $\ket{u_+}$. Otherwise, it outputs $\ket{u_-}$. 

As an example of the circuit that can be used to measure a single-qubit operator $U$, we consider $U \equiv Z$ . 

This implies that we can alternatively interpret the repetition code as measuring the product of $Z$-gates operating on the encoded qubits.

For that reason, let us define
\begin{align*}
    U_1=Z_1Z_2,\ U_2=Z_2Z_3
\end{align*}
In fact, an equivalent circuit for error detection, which incorporates $U_1$ and $U_2$ explicitly, can be constructed.

The operators $U_1$ and $U_2$ each has two distinct eigenvalues $\pm 1$. The measurement of the operators, for zero and one-qubit errors, yields
%TODO P28

Therefore, the measurement results contain the identical information as the measurement of $\ket{x}$ and $\ket{y}$. They are known as the \textbf{syndrome}, which can be used to identify and locate errors.

%TODO P29-30

\subsubsection{Toward Fault-Tolerance with the Surface Code}
In a surface code, qubits form a two-dimensional array. They are entangled into a randomly selected quiescent state resulting from the measurement of stabilizers.
%TODO P31-34

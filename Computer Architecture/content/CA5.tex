\newpage
\section{TLP Coherence}

\subsection{Thread-Level Parallelism}
TLP is identified by software or programmer and threads consist of (parallel) instructions

MIMD: multiple instruction streams and multiple data streams. 


multiprocessors: computers consisting of tightly coupled processors

Exploiting TLP:
\begin{itemize}
    \item Parallel processing
    \item Request-level parallelism
\end{itemize}

\subsection{Multiprocessor Architecture}
According to memory organization and interconnect strategy
Two classes:
\begin{itemize}
    \item symmetric/centralized shared-memory multiprocessors (SMP)
    \item distributed shared-memory multiprocessors (DMP)
\end{itemize}

\subsubsection{Centralized Shared Memory}
Share a single centralized memory; All processors have uniform latency from memory uniform memory access (UMA) multiprocessors


\subsubsection{Distributed Shared Memory}
Distributing memory among the nodes increases bandwidth \& reduces local-memory latency. 

NUMA: nonuniform memory access. access time depends on data word location in memory. 

\subsubsection{Hurdles of Parallel Processing}

\paragraph{Limited parallelism in programs} makes it difficult to achieve good speedups in any parallel processor (Amdahl's law)


\paragraph{Relatively high cost of communications} involves the large latency of remote access in a parallel processor


Improve Parallel Processing:
\begin{itemize}
    \item insufficient parallelism
    \item long-latency remote communication
\end{itemize}

\subsection{Centralized Shared Memory}

\subsubsection{Cache Coherence Problem}
\begin{itemize}
    \item Global state defined by main memory
    \item Local state defined by the individual
    caches, private to each processor core
\end{itemize}

A memory system is Coherent if any read of a data item returns the most recently written value of that data item. 

Two critical aspects:
\begin{itemize}
    \item coherence: defines what values can
    be returned by a read
    \item consistency: determines when a
    written value will be returned by a read
\end{itemize}

\subsubsection{Coherence Property}
A memory is coherent if
\begin{enumerate}
    \item A read by processor P to location X that
    follows a write by P to X (RAW), with no writes
    of X by another processor occurring
    between the write and the read by P,
    always returns the value written by P.
    \item A read by a processor to location X that
    follows a write by another processor to
    X (RAW) returns the written value if the read
    and the write are sufficiently separated
    in time and no other writes to X occur
    between the two accesses
    \item Write serialization:
    two writes (WAW) to the same location by any
    two processors are seen in the same
    order by all processors
\end{enumerate}

\subsubsection{Consistency} 
Memory consistency protocol. 

e.g, a write of X on one processor
precedes a read of X on another
processor by a very small time, it may
be impossible to ensure that the read
returns the value of the data written,
since the written data may not even
have left the processor at that point

\subsubsection{Cache Coherence Protocols}
\begin{itemize}
    \item Directory based: the sharing status of a particular block
    of physical memory is kept in one
    location, called directory
    \item Snooping: every cache that has a copy of the data
    from a block of physical memory could
    track the sharing status of the block
\end{itemize}


\subsubsection{MSI protocol}
Snooping Coherence Protocols: 
\begin{itemize}
    \item Write invalidate protocol: invalidate other copies on a write
    \item Write update/broadcast protocol
\end{itemize}

Implement Write Invalidate Protocol: MSI protocol: three block states
\begin{itemize}
    \item Invalid
    \item Shared: 
    indicates that the block in the private
    cache is potentially shared
    \item Modified: 
    indicates that the block has been
    updated in the private cache;
    implies that the block is exclusive
\end{itemize}

%TODO 图 11.69-11.71 与表格 11.64-67


MSI Extensions: 

MESI
\begin{itemize}
    \item exclusive: indicates when a cache
    block is resident only in a single cache
    but is clean (after initial read)
\end{itemize}

MOESI
\begin{itemize}
    \item owned: indicates that the associated
    block is owned by that cache and out-of-date in memory
\end{itemize}

%TODO 图11.81

MESIF
\begin{itemize}
    \item forward: designates which sharing
    processor (among ones with the same
    shared-state data block) should respond
    to a request
\end{itemize}

\subsubsection{Coherence Miss}
True sharing miss:
\begin{enumerate}
    \item first write by a processor to
    a shared cache block causes an
    invalidation to establish ownership of
    that block;
    \item another processor reads a
    modified word in that cache block;
\end{enumerate}

False sharing miss:
a single valid bit per cache block;
occurs when a block is invalidated (and
a subsequent reference causes a miss)
because some word in the block, other
than the one being read, is written into


\subsection{Distributed Shared Memory}
A directory is added to each node;
Each directory tracks the caches that share the
memory addresses of the portion of memory in
the node; 

need not broadcast on every cache miss as in
snooping-based coherence protocol

\subsubsection{Directory-based Cache Coherence Protocol}
Common cache states
\begin{itemize}
    \item Shared: 
    one or more nodes have the block cached,
    and the value in memory is up to date (as
    well as in all the caches)
    \item Uncached: 
    no node has a copy of the cache block
    \item Modified: 
    exactly one node has a copy of the cache
    block, and it has written the block, so the
    memory copy is out of date
\end{itemize}

%TODO 表格 11.103-117 图 119-120
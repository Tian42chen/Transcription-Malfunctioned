% 非考试重点
\newpage
\section{DLP Architecture}%TODO wip
\subsection{Data-Level Parallelism (DLP)}
\begin{itemize}
    \item Vector Architecture
    \item Multimedia SIMD
    \item GPU
\end{itemize}

\subsection{Vector Architecture}
\begin{enumerate}
    \item Grab sets of data elements scattered about memory. 
    \item Place them in large sequential reg files. 
    \item Place them in large sequential reg files. 
    \item Disperse the results back into memory
\end{enumerate}

A single instruction works on vector of data, which results in dozens of register-register operations on independent data elements. 


\subsubsection{RV64V}


\subsubsection{RV64V Vector Instructions}

\subsubsection{Dynamic Register Typing}
Omitted in instruction encoding. Associated a data type and data size with each vector register. 


Enable programs to disable unused vector registers. 

\subsubsection{DAXPY Example}
Double precision a $\times$ X $+$ Y
\begin{itemize}
    \item X and Y have 32 elements
    \item x5 and x6 hold the starting addresses
    of X and Y, respectively
\end{itemize}

\begin{code}
    \caption{RV64V code}
    \begin{minted}{asm}
        fld f0.a            #
        addi x28, x5, #256  # Last address to load
    Loop:                   #
        fld f1, 0(x5)       # 
        fmul.d f1, f1, f0   # 
        fld f2, 0(x6)       # 
        fadd.d f2, f2, f1   # 
        fsd f2, 0(x6)       # 
        addi x5, x5, #8     # 
        addi x6, x6, #8     # 
        bne x28, x5 Loop    # 
    \end{minted}
\end{code}

\begin{code}
    \caption{RV64V code: double-precision}
    \begin{minted}{asm}
        vsetdcfg 4*FP64     # Enable 4 DP FP vregs
        fld f0, a           # 
        vld v0, x5          # 
        vmul v1, v0, f0     # 
        vld v2, x6          # 
        vadd v3, v1, v2     # 
        vst v3, x6          # 
        vdisable            # Disable vector regs
    \end{minted}
\end{code}
no loop-carried dependences

Chaining: 
\begin{itemize}
    \item forwarding of element-dependent operations, in that dependent operations are chainedtogether
    \item scalar and vector registers
    generate interim results which can be
    used immediately, without additional
    memory references
\end{itemize}
Flexible Chaining: allow a vector
instruction to chain to essentially any
other active vector instruction

\subsubsection{Vector Execution Time}
For a sequence of vector operations
\begin{itemize}
    \item Convoy
    \subitem the set of vector instructions that could
    potentially execute together
    \subitem co-convoy instructions must not contain
    any structural hazards
    \subitem RAW is allowed in the same convoy
    \item Chime: 
    \subitem the unit of time taken to execute the
    convoy
\end{itemize}
Cross-convoy instructions are serialized

e.g. %TODO 39

\subsection{optimize vector architecture}
\subsubsection{Multiple Lanes}
use multiple functional units to
improve vector performance;
process several elements per
clock cycle

vector register memory is divided across the lanes

\subsubsection{Vector-Length Registers}
Vector length is often unknown at
compile time

Vector-length register (vl) controls the
length of any vector operation,
including a vector load or store

The value in vl cannot be greater than
the maximum vector length (mvl)

e.g. %TODO 52

\subsection{handle IF in loops}
\subsubsection{Predicate Registers}
to handle IF in loops

Vector-mask control: 
use predicate registers to hold the mask
and essentially provide conditional
execution of each element operation

\subsubsection{Memory Banks}


\subsubsection{Stride}

\subsubsection{Gather-Scatter}


\newpage
\section{DLP Exploitation}

\subsection{Loop-Level Parallelism}
\begin{itemize}
    \item in-iteration dependence
    \item cross-iteration/Loop-carried dependences
    \item No circular loop-carried dependence:
    self dependence, mutual dependence;
\end{itemize}

e.g. 

\subsection{find dependences}
\subsubsection{Index Affinity}

\subsubsection{Dependence Conditions}

\subsubsection{GCD Test}
GCD (Greatest Common Divisor) Test:
$a, b, c$ and $d$ are all constants, i.e. ai+b and ci+d. If a loop-carried dependence exists,
\begin{align*}
    (d-b)\% gcd(c,a)=0
\end{align*}

\subsection{Dependency Types}


\subsection{Dependency Elimination}
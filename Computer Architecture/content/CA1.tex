\newpage
\section{Fundamentals-Basics}

%TODO 补图?
\subsection{RISC Architecture}
Reduced Instruction Set Computer

Present two critical perf techniques: 

\begin{enumerate}
    \item instruction-level parallelism (pipelining and multiple instruction issue)
    \item caching
\end{enumerate}

\subsubsection{Pipelining}
Divide instruction execution into stages

Overlap execution of multi-instruction (CPI>1)


\subsubsection{Multiple Instruction Issue}
Deploy multiple datapaths

Complete more than one instruction per clock cycle (CPI may <1)

\subsubsection{Caching}
faster temporary storage


\subsubsection{Dennard Scaling}
Power density is constant for a given area of silicon even as you increase the number of transistors because of smaller dimensions of each transistor

失效了

\subsubsection{Moore's Law}
The nubmer of transistors per chip would double every year (every two years later). 

\subsubsection{Multi-Core Processor}
Multiple efficient processors. From instruction-level parallelism to data-level and thread-level parallelism. 

\subsubsection{Amdahl's Law}
\begin{align*}
    111
\end{align*}

\subsection{5 Classes of Computer}
\begin{enumerate}
    \item PMD: Personal Mobile Device
    \item Desktop Computing
    \item Servers
    \item Clusters/WSCs
    \item Embedded Computers
\end{enumerate}



\subsection{What makes computers fast}
\subsubsection{Parallelism}
application parallelism and hardware parallelism

\paragraph{Application Parallelism}


\paragraph{Hardware Parallelism}
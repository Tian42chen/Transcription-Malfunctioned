\newpage
\section{Fundamentals-Basics}

%TODO 补图?
\subsection{RISC Architecture}
Reduced Instruction Set Computer

Present two critical perf techniques: 

\begin{enumerate}
    \item instruction-level parallelism (pipelining and multiple instruction issue)
    \item caching
\end{enumerate}

\subsubsection{Pipelining}
Divide instruction execution into stages

Overlap execution of multi-instruction (CPI>1)


\subsubsection{Multiple Instruction Issue}
Deploy multiple datapaths

Complete more than one instruction per clock cycle (CPI may <1)

\subsubsection{Caching}
faster temporary storage


\subsubsection{Dennard Scaling}
Power density is constant for a given area of silicon even as you increase the number of transistors because of smaller dimensions of each transistor

失效了

\subsubsection{Moore's Law}
The nubmer of transistors per chip would double every year (every two years later). 

\subsubsection{Multi-Core Processor}
Multiple efficient processors. From instruction-level parallelism to data-level and thread-level parallelism. 

\subsubsection{Amdahl's Law}
Make common case fast
\begin{align*}
    1&=\text{Fraction}_{\text{enhanced}}+\text{Fraction}_{\text{last}}\\
    \text{Speedup}_{\text{overall}}&=\frac{\text{Execution tmie}_{\text{old}}}{\text{Execution time}_{\text{new}}}\\
    &=\frac{1}{\text{Fraction}_{\text{last}}+\frac{\text{Fraction}_{\text{enhanced}}}{\text{Speedup}_{\text{enhanced}}}}
\end{align*}

\subsection{5 Classes of Computer}
\begin{enumerate}
    \item PMD: Personal Mobile Device
    \item Desktop Computing
    \item Servers
    \item Clusters/WSCs
    \item Embedded Computers
\end{enumerate}

\subsection{Parallelism}
application parallelism and hardware parallelism

\subsubsection{Application Parallelism}
\begin{itemize}
    \item DLP: Data-Level Parallelism
    \item TLP: Task-Level Parallelism
\end{itemize}

\subsubsection{Hardware Parallelism}
four ways:
\begin{itemize}\small
    \item ILP: Instruction-Level Parallelism
    \subitem exploit data-level parallelism
    \begin{itemize}
        \item pipelining
        \subitem divide a task to steps;
        \subitem simultaneously run different steps of different tasks
        \item speculative execution
        \subitem do some work in advance;
        \subitem prevent delay when the work is needed
    \end{itemize}
    \item Vector Architectures, GPUs, and Multimedia Instruction Sets(MM)%TODO P61 图
    \subitem exploit data-level parallelism;
    \subitem apply a single instruction to a collection of data in parallel;
    \item TLP: Thread-Level Parallelism % P63
    \subitem exploits either DLP or TLP,
    \subitem in a tightly coupled hardware model that allows for interaction among parallel threads;
    \item RLP: Request-Level Parallelism P65
    \subitem exploits parallelism among largely decoupled tasks specified by the programmer or the OS
\end{itemize}

\subsubsection{Classes of Parallel Architectures}
\textbf{S}ingle/\textbf{M}ultiple \textbf{I}nstruction/\textbf{D}ata stream
\begin{itemize}
    \item SISD: Exploit instruction-level parallelism
    \item SIMD: The same instruction is executed by multiple processors using different data streams. Exploits data-level parallelism.
    \item MISD: hard to exploit data-level parallelism
    \item MIMD: Each processor fetches its own instructions and operates on its own data. Exploits task-level parallelism
\end{itemize}

\subsection{Instruction Set Architecture (ISA)}
7 Dimensions of ISA:
\begin{itemize}
    \item Class of ISA
    \item Memory addressing
    \item Addressing modes: %TODO P103
    \item Types and sizes of operands
    \item Operations
    \begin{itemize}
        \item Data Transfer
        \item Arithmetic/Logical
        \item Control
        \item Floating point
    \end{itemize}
    \item Control flow instructions
    \item Encoding an ISA
\end{itemize}
remember RISC-V Register Naming %TODO P90

\section{Fundamentals-Performance}
\subsection{Trend}
energy workload %TODO P27

daynamic Energy %TODO P31

daynamic Power %TODO P32

static Power %TODO P45

Dies per Wafer %TODO 52

Cost of Die %TODO 53

Die Yield %TODO 54

Cost of Integrated Circuit %TODO 57

Price %TODO 60

Transient/permanent faults become more commonplace
\subsection{Dependability}
SLA: service level agreements

SLO: service level objectives

System states: up or down

Service states %TODO P68

\begin{itemize}%TODO 72-73
    \item Failure: 仅当 actual behavior 偏离 actual behavior 时发生. 
    \item Error
    \item Fault
\end{itemize}

\subsubsection{Measures of Dependability}
Module reliability: A measure of continuous service accomplishment (or of the time to failure) from a reference initial instant. 

\begin{itemize}%TODO P82
    \item MTTF: mean time to failure
    \item MTTR: mean time to repair
    \item MTBF: mean time between failures
    \item FIT: failures per billion hours
\end{itemize}
MTBF = MTTF + MTTR


Module availability=%TODO P87

\subsubsection{Dependability via Redundancy}


\subsection{Disk Arrays - RAID}

\subsection{Measure Performance}





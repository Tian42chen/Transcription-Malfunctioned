\newpage
\section{概率论基本概念}
\subsection{事件的关系与运算}
\subsubsection{随机试验}
对随机现象的观察、记录、实验统称为\highlight{随机试验}. 它具有以下特性: 
\begin{enumerate}
    \item 可以在相同条件下重复进行; 
    \item 事先知道可能出现的结果$(\ge 2)$; 
    \item 进行试验前并不知道哪个试验结果会发生.  
\end{enumerate}

\subsubsection{样本空间}
\begin{definition}
    随机试验E的所有结果构成的集合称为$E$的\highlight{样本空间}, 记为$S=\{e\}$, 称$S$中的元素$e$为\highlight{样本点}, 一个元素的单点集称为\highlight{基本事件}.
\end{definition}

\subsubsection{随机事件}
一般我们称$S$的子集$A$为$E$的\highlight{随机事件$A$}, 简称\highlight{事件$A$}.当且仅当$A$所包含的一个样本点发生称\highlight{事件$A$发生}. 

\subsubsection{运算与关系}
和、差、积、逆、包含、不相容

\subsection{概率定义与性质}
\subsubsection{频率}
\begin{definition}
    $f_n(A)=\frac{n_A}{n}$, 在$n$次中$A$发生$n_A$次. 
\end{definition}

性质: 
\begin{enumerate}
    \item $0\le f_n(A) \le 1$
    \item $f_n(S)=1$
    \item 若$A_1,A_2,\dots,A_k$两两不相容, 则
    \begin{align*}
        f_n\left(\bigcup_{i=1}^k A_i\right)=\sum_{i=1}^k f_n(A_i)
    \end{align*}
\end{enumerate}

\subsubsection{概率}
\begin{definition}
    对样本空间$S$中任一事件$A$, 定义一个实数$P(A)$, 满足以下三条公理:
    \begin{enumerate}
        \item 非负性:$P(A)\ge 0$
        \item 规范性:$P(S)=1$
        \item 可列可加性:若$A_1,A_2,\dots,A_k,\dots$两两不相容, 则
        \begin{align*}
            P\left(\bigcup_{i=1}^{\infty}A_i\right)=\sum_{i=1}^{\infty}P(A_i)
        \end{align*}
    \end{enumerate}
称$P(A)$为事件$A$的\highlight{概率}. 
\end{definition}

性质:
\begin{enumerate}
    \item $P(\phi)=0$
    \item $A_1,A_2,\dots,A_n$, $A_i A_j=\phi, i\ne j$
    \begin{align*}
        P\left(\bigcup_{i=1}^{n}A_i\right)=\sum_{i=1}^{n}P(A_i)
    \end{align*}
    \item $P(A)=1-P(\bar{A})$
    \item $P(B-A)=P(B)-P(AB)=P(B\bar{A})$
    \item $P(A\cup B)=P(A)+P(B)-P(AB)$
    \begin{align*}
        P\left(\bigcup_{i=1}^n A_i\right)=&\sum_{i=1}^n P(A_i)-\sum_{1\le i<j\le n}P(A_i A_j)\\
        &+\sum_{1\le i<j<k\le n}P(A_i A_j A_k)+\cdots\\
        &+(-1)^{n-1}P(A_1 A_2 \cdots A_n)
    \end{align*}
    \item $P(AB)=P(A)P(B|A)$ 若 $P(A)>0$
    \begin{align*}
        P(A_1\cdots A_n)=P(A_1)P(A_2|A_1)\cdots P(A_n|A_1\cdots A_{n-1})
    \end{align*}
\end{enumerate}

\subsection{等可能概型}
\begin{definition}
    若试验$E$满足:
    \begin{enumerate}
        \item S中样本点有限(有限性)
        \item 出现每一样本点的概率相等(等可能性)
    \end{enumerate}
    称这种试验为等可能概型(或古典概型).  
    \[ P(A)=\frac{A \text{所包含的样本点数}}{S \text{中的样本点数}} \]
\end{definition}

\subsection{条件概率}
\begin{definition}
    \[ P(B|A)=\frac{P(AB)}{P(A)}, \, P(A)>0 \]
\end{definition}

\subsubsection{全概率公式与Bayes公式}
\begin{definition}
    称$B_1,B_2,\dots,B_n$为$S$的一个划分, 若
    \begin{enumerate}
        \item 不漏 $B_1 \cup B_2 \cup \cdots \cup B_n =S$
        \item 不重 $B_i B_j=\phi, \, i\ne j $
    \end{enumerate}
\end{definition}

\begin{theorem}
    设$B_1,B_2,\dots,B_n$为样本空间$S$的一个划分, $P(B_i)>0, \, i=1,2,\dots,n$, 则称
    \[ P(A)=\sum_{j=1}^n P(B_j)\cdot P(A|B_j)\]
    为\highlight{全概率公式}.
\end{theorem}

\begin{theorem}
    接全概率公式的条件, 且$P(A)>0$,则
    \[ P(B_i|A)=\frac{P(B_i)P(A|B_i)}{\sum_{j=1}^n P(B_j)\cdot P(A|B_j)}\]
    称为\highlight{Bayes公式}.
\end{theorem}

\subsection{事件独立性与独立试验}
\begin{definition}
    设$A, B$为两随机事件, 如果 $P(AB)=P(A)P(B)$, 则称\highlight{$A$与$B$ 相互独立}. 若$P(A)\ne 0, P(B)\ne 0$, 则
    \begin{align*}
        P(AB)=P(A)P(B)&\Longleftrightarrow P(A|B)=P(A)\\
        &\Longleftrightarrow P(B|A)=P(B)
    \end{align*}
\end{definition}
\begin{definition}
    设$A_1,A_2,\dots,A_n$为$n$个随机时间, 若对$2\le k \le n$, 均有
    \begin{align*}
        P(A_{i_1}A_{i_2}\cdots A_{i_n})=\prod_{j=1}^k P(A_{i_j})
    \end{align*}
    则称$A_1,A_2,\dots,A_n$相互独立. 
\end{definition}
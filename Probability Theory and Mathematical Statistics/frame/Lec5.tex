\newpage
\section{大数定律和中心极限定理}
\subsection{依概率收敛}
\begin{definition}
    $Y_1,\dots,Y_n$, $\exists c\in R, \forall \epsilon>0$,
    \[\lim_{n\rightarrow +\infty}P\left\{\left| Y_n-c \right|\ge \epsilon \right\}=0 \]
    称$\left\{Y_n,n\ge 1\right\}$依概率收敛于$c$, 记$Y_n \overset{P}{\longrightarrow}c, (n\rightarrow +\infty)$. 
\end{definition}

性质: 若$X_n \overset{P}{\longrightarrow}a, Y_n \overset{P}{\longrightarrow}b$, $g\in \mathbb{C}(a,b)$,
\begin{align*}
    g(X_n,Y_n) \overset{P}{\longrightarrow} g(a,b)
\end{align*}

\subsection{不等式}
\subsubsection{马尔可夫不等式}
\begin{theorem}
    对随机变量$Y$, $\exists k$ 阶矩 $(k\ge 1)$, $\forall \epsilon >0$
    \begin{align*}
        P\left\{|Y|\ge \epsilon\right\}\le& \frac{E\left(|Y|^k\right)}{\epsilon^k}\\
        \text{or }P\left\{|Y|<\epsilon\right\}\ge & 1-\frac{E\left(|Y|^k\right)}{\epsilon^k}
    \end{align*}
\end{theorem}
\subsubsection{切比雪夫不等式}
\begin{theorem}
    对随机变量$X$, $\exists Var(x)$, $\forall \epsilon >0$
    \begin{align*}
        P\left\{|X-E(X)|\ge \epsilon\right\}\le& \frac{Var(x)}{\epsilon^2}\\
        P\left\{|X-E(X)|< \epsilon\right\}\ge& 1-\frac{Var(x)}{\epsilon^2}
    \end{align*}
\end{theorem}

\subsection{大数定律}
\subsubsection{弱大数定律}
\begin{theorem}
    随机变量序列$Y_1,\dots,Y_n,\dots$, 若$\exists \left\{c_n, n\ge 1\right\}$, 则当$n\rightarrow +\infty$,  
    \begin{align*}
        \frac{1}{n}\sum_{i=1}^n Y_i-c_n\overset{P}{\longrightarrow}0
    \end{align*}
    即$\forall \epsilon>0$, 
    \begin{align*}
        \lim_{n\rightarrow \infty}P\left\{ \left| \frac{1}{n}\sum_{i=1}^n Y_i-c_n \right|\ge \epsilon \right\}=0
    \end{align*}
    特别地, 当$c_n\equiv c, n=1,2,\dots$, $n\rightarrow +\infty$, 
    \begin{align*}
        \frac{1}{n}\sum_{i=1}^n Y_i\overset{P}{\longrightarrow}c
    \end{align*}
\end{theorem}

\subsubsection{切比雪夫大数定律}
\begin{theorem}
    $X_1,X_2,\dots,X_n$相互独立, 且$\exists E(X_i)=\mu$, \\$Var(X_i)=\sigma^2$, 则$n\rightarrow +\infty$, 
    \begin{align*}
        \frac{1}{n}\sum_{i=1}^n X_i \overset{P}{\longrightarrow} \mu
    \end{align*}
\end{theorem}

\subsubsection{辛钦大数定律}
\begin{theorem}
    $X_1,\dots, X_n$独立同分布, 且$\exists E(X_i)=\mu$, 则当$n\rightarrow +\infty$, 
    \begin{align*}
        \frac{1}{n}\sum_{i=1}^n X_i \overset{P}{\longrightarrow} \mu
    \end{align*}
    若$h(x)\in \mathbb{C}, \, \exists E(h(X_1))=a$, 则当$n\rightarrow +\infty$, 
    \begin{align*}
        \frac{1}{n}\sum_{i=1}^n h(x_i) \overset{P}{\longrightarrow} a
    \end{align*}
\end{theorem}

\subsubsection{贝努里大数定律}
\begin{theorem}
    $n_A$为$n$重贝努里实验中$A$发生次数, $P(A)=p$, 则当$n\rightarrow +\infty$, 
    \begin{align*}
        \frac{n_A}{n}\overset{P}{\longrightarrow}p
    \end{align*}
\end{theorem}

\subsection{中心极限定律}
\subsubsection{林德贝格-列维中心极限定理 (独立同分布)}
\begin{theorem}
    $X_1,X_2,\dots,X_n$独立同分布, $E(X_i)=\mu$,\\  $Var(X_i)=\sigma^2$, 当$n$足够大时 $(n>50)$, 
    \begin{align*}
        \sum_{i=1}^n X_i \sim& N(n\mu,n\sigma^2)\\
        \frac{1}{n}\sum_{i=1}^n X_i \sim& N(\mu,\frac{\sigma^2}{n})
    \end{align*}
\end{theorem}

\subsubsection{棣莫弗—拉普拉斯中心极限定理 (二项分布)}
\begin{theorem}
    $n_A$为$n$重贝努里实验中$A$发生次数, $P(A)=p$, 当$n$足够大时, 
    \begin{align*}
        B(n,p)\sim N\left(np,np(1-p)\right)
    \end{align*}
\end{theorem}
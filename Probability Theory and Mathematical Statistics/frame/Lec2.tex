\newpage
\section{随机变量及分布函数}

\subsection{随机变量}
\begin{definition}
    设随机试验的样本空间$S=\{e\}$ , 若$X=X(e)$为定义在样本空间$S$上的实值单值函数, 则称$X=X(e)$为随机变量. 
\end{definition}

\subsection{分布函数}
\begin{definition}
    随机变量$X$的分布函数: 
    \[ \forall x \in R, \, F(x)=P(X\le x) \]
\end{definition}

性质: 
\begin{enumerate}
    \item $0\le F(x) \le 1$
    \item $F(x)$单调不减, 且$F(-\infty )=0, F(+\infty)=1$
    \item $F(x)$右连续, 即$F(x+0)=F(x)$
    \item $F(x)-F(x-0)=P(X=x)$
\end{enumerate}

\subsection{离散型随机变量}
\begin{definition}
    取值至多可数的随机变量为\highlight{离散型的随机变量}. 
    \begin{align*}
        \text{概率分布律性质: }& p\ge 0,\,\sum_{i=1}^{\infty}p_i=1\\
        \text{分布函数: }& F(x)=\sum_{x_k\le x} p_k
    \end{align*}
\end{definition}

\subsubsection{0-1分布 \texorpdfstring{$X\sim B(1,p)$}.}
\begin{table}[H]
    \centering
    \begin{tabular}[c]{|c|cc|}\hline
        X & 0 & 1\\ \hline
        P & $1-p$ & $p$\\ \hline
    \end{tabular}
    % \caption{0-1}
\end{table}

\begin{align*}
    F(x)=P\left\{X\le x\right\}=\left\{ \begin{array}{lc}
        0 &x<0 \\ 1-p & 0\le x <1 \\ 1 & x\ge 1
    \end{array} \right. \\
\end{align*}
\begin{align*}
    E(X)=p,\, Var(X)=p(1-p)
\end{align*}

\subsubsection{n重贝努里试验与二项分布 \texorpdfstring{$X\sim B(n,p)$}.}
进行$n$次独立重复观测, 每次$A$或$\bar{A}$发生, $P(A)=p$, $X$表示$A$发生的次数. 

\begin{align*}
    P(X=k)=C_n^k p^k (1-p)^{n-k},\, k=0,1,\dots,n
\end{align*}
\begin{align*}
    E(x)=np,\, Var(X)=np(1-p)
\end{align*}


\subsubsection{泊松分布 \texorpdfstring{$X\sim P(\lambda)$}.}
\begin{align*}
    P(X=k)=\frac{e^{-\lambda}\lambda^k}{k!}, \, k=0,1,\dots
\end{align*}
\begin{align*}
    E(X)=Var(X)=\lambda
\end{align*}

泊松定理:
\begin{theorem}
    当$n>10, p<0.1$时, 
    \begin{align*}
        C_n^k p^k(1-p)^{n-k} \approx \frac{e^{-\lambda}\lambda^k}{k!}, \, \lambda=np
    \end{align*}
\end{theorem}

\subsection{连续型随机变量}
\begin{definition}
    对于随机变量$X$的分布函数$F(x)$, 若对$\forall x \in R$, $\exists f(x)\ge 0$, 
    \begin{align*}
        F(x)=\int_{-\infty}^{x}f(t)\,\mathrm{d}t
    \end{align*}
    则称$X$为\highlight{连续型随机变量}, 其中$f(x)$称为$X$的\highlight{概率密度函数}, 简称\highlight{密度函数}. 
\end{definition}

性质:
\begin{enumerate}
    \item $f(x)\ge 0$
    \item $\int_{-\infty}^{\infty}f(x)\,\mathrm{d}x=1$
    \item $P(a\le X\le b)=P(a<X<b)=F(b)-F(a)=\int_{b}^{a}f(x)\,\mathrm{d}x$
\end{enumerate}

\subsubsection{均匀分布 \texorpdfstring{$X\sim U(a,b)$}.}
\begin{align*}
    f(x)=&\left\{\begin{array}{lc}
        \frac{1}{b-a} & x\in(a,b)\\0 & otherwise
    \end{array} \right.\\
    F(x)=&\left\{\begin{array}{lc}
        0 & x<a\\ \frac{x-a}{b-a} & a\le x<b \\ 1 & x\ge b
    \end{array} \right.
\end{align*}
\begin{align*}
    E(X)=\frac{a+b}{2}, \, Var(X)=\frac{(b-a)^2}{12}
\end{align*}
\begin{align*}
    a<c<d<b, \, P(c\le X \le d)=\frac{d-c}{b-a}
\end{align*}

\subsubsection{正态分布 \texorpdfstring{$X\sim N(\mu,\sigma^2)$}.}
\begin{align*}
    f(x)=\frac{1}{\sqrt{2\pi} \sigma}e^{-\frac{(x-\mu)^2}{2\sigma^2}}, \, -\infty<x<\infty
\end{align*}
\begin{align*}
    E(X)=\mu,\,Var(X)=\sigma^2
\end{align*}

标准正态分布$N(0,1)$:
\begin{align*}
    \psi(x)=&\frac{1}{\sqrt{2\pi}}e^{-\frac{x^2}{2}}\\
    \Phi(x)=&\int_{-\infty}^x \frac{1}{\sqrt{2\pi}}e^{-\frac{t^2}{2}}\,\mathrm{d}t
\end{align*}

性质: 
\begin{enumerate}
    \item $\Phi(-x)+\Phi(x)=1,\,\Phi(0)=\frac{1}{2}$
    \item $\Phi(z_{\alpha})=1-\alpha, \, z_{1-\alpha}=-z_{\alpha}$
    \item 当$X\sim N(\mu,\sigma^2)$时: 令$t=\frac{x-\mu}{\sigma} \sim N(0,1)$, 
    \begin{align*}
        P(x\le b)=\Phi(\frac{b-\mu}{\sigma})
    \end{align*}
    \item $aX+b \sim N(a\mu+b, a^2\sigma^2)$
\end{enumerate}

\subsubsection{指数分布 \texorpdfstring{$X\sim Exp(\lambda)$}.}
\begin{align*}
    f(x)=&\left\{ \begin{array}{lc}
        \lambda e^{-\lambda x} & x>0 \\ 0 & x\le 0
    \end{array} \right.\\
    F(x)=&\left\{\begin{array}{lc}
        1-e^{-\lambda x} & x>0 \\ 0 & x\le 0
    \end{array} \right. 
\end{align*}
\begin{align*}
    E(X)=\frac{1}{\lambda},\, Var(X)=\frac{1}{\lambda^2}
\end{align*}

无记忆性: 
\begin{align*}
    s>0,t>0,\, P(X>s+t|X>s)=P(X>t)=e^{-\lambda t}
\end{align*}

\subsection{随机变量函数的分布}
已知$X$的概率分布, $Y=g(X)$, 求$Y$的概率分布. 
\begin{enumerate}
    \item 若Y为离散型随机变量, 则先写出$Y$的可能取值: $y_1,y_2,\dots,y_j,\dots$, 再找出$(Y=y_i)$的等价事件$(X\in D_j)$, 得 $P(Y=y_i)=P(X\in D_j)$. 
    \item 若Y为连续型随机变量, 则先写出$Y$的概率分布函数: $F_Y(y)=P(Y\le y)$, 找出$(Y\le y)$的等价事件$(X\in D_y)$, 得$F_Y(y)=P(X\in D_y)$; 再求出$Y$的概率密度函数$f_Y(y)$.
\end{enumerate}

\begin{definition}
    设$X\sim f_X(x)$, $-\infty < x < \infty$, $y=g(x)$是严增(减)可微函数, 即 $g'(x)>0$(或$g'(x)<0$). $Y=g(X)$,  则$Y$具有概率密度函数为
    \begin{align*}
        f_Y(y)=\left\{\begin{array}{lc}
            f_X\left(h(y)\right)\cdot \left| h'(y) \right| & \alpha<y<\beta\\ 0 & otherwise
        \end{array} \right.
    \end{align*}
    这里$(\alpha,\beta)$是$Y$的取值范围, $h$是$g$的反函数, $h(y)=x\Longleftrightarrow y=g(x)$.
\end{definition}
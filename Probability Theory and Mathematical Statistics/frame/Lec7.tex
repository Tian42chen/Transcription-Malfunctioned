\newpage
\section{参数估计}

\subsection{矩估计}
总体$X$有m个未知参数$\theta_1,\dots,\theta_m$, \\$\exists m$阶矩 $\mu_1,\dots,\mu_m$
\begin{enumerate}
    \item 计算 $\mu_k=E(X^k)=g_k(\theta_1,\dots,\theta_m),\,k=1,\dots,m$.
    \item 求反函数, 得$\theta_k=h_k(\mu_1,\dots,\mu_m),\,k=1,\dots,m$.
    \item 以样本各阶矩$A_1,\dots,A_m$代替\\总体各阶矩$\theta_1,\dots,\theta_m$, 得各参数的矩估计
    \begin{align*}
        \hat{\theta}=h_k(A_1,\dots,A_m),\,k=1,\dots,m
    \end{align*}
\end{enumerate}

\subsection{极大似然估计}
设总体$X$的分布律为$p(x;\theta)$ (或密度函数$f(x;\theta)$), $\theta\in \Theta$. 从总体$X$中取得样本$X_1,\dots,X_n$, 其观察值为$x_1,\dots,x_n$. 似然函数 $L(\theta)=\prod_{i=1}^n p(x_i;\theta)$ (或$L(\theta)=\prod_{i=1}^n f(x_i;\theta)$). 
\begin{align*}
    \text{极大似然原理 }L\left(\hat{\theta}(x_1,\dots,x_n)\right)=\max_{\theta\in \Theta}L(\theta)
\end{align*}
$\hat{\theta}(x_1,\dots,x_n)$称为$\theta$的极大似然估计值, 相应统计量$\hat{\theta}(X_1,\dots,X_n)$称为$\theta$的极大似然估计量(MLE). 

求解:
\begin{enumerate}
    \item 令$\ln L(\theta)=l(\theta)$, 称为对数似然函数, 再令
    \begin{align*}
        \left.\frac{\partial l(\theta)}{\partial \theta_i}\right|_{\hat{\theta}_i, 1\le i \le k}=0
    \end{align*}
    解得$\hat{\theta}_i$, $i=1,2,\dots,k$. 
    \item 若$L(\theta)$关于某个$\theta_i$单调增(减),
    \begin{align*}
        \theta_i \le (\ge) \hat{\theta}_i(x_1,\dots,x_n)
    \end{align*}
    此时$\hat{\theta}(x_1,\dots,x_n)$为$\theta$的极大似然估计值, 相应统计量$\hat{\theta}(X_1,\dots,X_n)$为$\theta$的极大似然估计量. 
    \item 若$\hat{\theta}(X_1,\dots,X_n)$为$\theta$的极大似然估计量, 则$g(\theta)$的极大似然估计量为$g\left(\hat{\theta}(X_1,\dots,X_n)\right)$. 
\end{enumerate}

\subsection{估计量评选准则}
\subsubsection{无偏性准则}
\begin{definition}
    若参数$\theta$的估计量$\hat{\theta}=\hat{\theta}(X_1,\dots,X_n)$满足
    \begin{align*}
        E(\hat{\theta})=\theta
    \end{align*}
    则称$\hat{\theta}$是$\theta$的无偏估计量. 
\end{definition}
\subsubsection{有效性准则}
\begin{definition}
    $\hat{\theta}_1, \hat{\theta}_2$是$\theta$的两个无偏估计量, 如果$\forall \theta\in \Theta$,
    \begin{align*}
        Var(\hat{\theta}_1)\le Var(\hat{\theta}_2)
    \end{align*}
    且不等号至少对某一$\theta$成立, 则称$\hat{\theta}_1$比$\hat{\theta}_2$有效.
\end{definition}
\subsubsection{均方误差准则}
\begin{definition}
    $Mse(\hat{\theta})=E\left(\hat{\theta}-\theta\right)^2$称为$\hat{\theta}$关于$\theta$的均方误差. $\hat{\theta}_1, \hat{\theta}_2$是$\theta$的两个估计量, 如果$\forall \theta\in \Theta$,
    \begin{align*}
        Mse(\hat{\theta}_1)\le Mse(\hat{\theta}_2) 
    \end{align*}
    且不等号至少对某一$\theta$成立,则称$\hat{\theta}_1$优于$\hat{\theta}_2$. 
\end{definition}
\subsubsection{相合性准则}
\begin{definition}
    设$\hat{\theta}_n=\hat{\theta}(X_1,\dots,X_n)$为参数$\theta$的估计量, 若$\forall \theta\in\Theta$, 当$n\rightarrow +\infty$, 
    \begin{align*}
        \hat{\theta}_n\overset{P}{\longrightarrow}\theta
    \end{align*}
    即 $\forall \epsilon>0$,
    \begin{align*}
        \lim_{n\rightarrow +\infty}P\left\{\left|\hat{\theta}_n-\theta\right|\ge \epsilon \right\}=0
    \end{align*}
    则称$\hat{\theta}_n$为$\theta$的 \highlight{相合估计量} 或 \highlight{一致估计量}. 
\end{definition}


\subsection{置信区间}
\begin{definition}
    设总体$X$的分布函数$F(x;\theta)$, $\theta$未知.对给定值$\alpha(0<\alpha<1)$, 有两个统计量 $\theta_L=\theta_L(X_1,\dots,X_n)$,$\theta_U=\theta_U(X_1,\dots,X_n)$, 使得
    \begin{align*}
        P\left\{ \theta_L(X_1,\dots,X_n)<\theta<\theta_U(X_1,\dots,X_n) \right\}\ge 1-\alpha
    \end{align*}
    称$\left(\theta_L,\theta_U\right)$是$\theta$的\highlight{置信水平}为$1-\alpha$的\highlight{双侧置信区间}; $\theta_L,\theta_U$分别为\highlight{双侧置信下限}和\highlight{双侧置信上限}. 

    若
    \begin{align*}
        P\left\{ \theta_L(X_1,\dots,X_n)<\theta \right\}\ge 1-\alpha
    \end{align*}
    则称$\theta_L$是$\theta$的置信水平为$1-\alpha$的\highlight{单侧置信下限}. 

    若
    \begin{align*}
        P\left\{ \theta<\theta_U(X_1,\dots,X_n) \right\}\ge 1-\alpha
    \end{align*}
    则称$\theta_U$是$\theta$的置信水平为$1-\alpha$的\highlight{单侧置信上限}. 
\end{definition}


\subsection{正态总体均值, 方差的置信区间与单侧置信限}

\begin{table*}[htb]
    \centering
    \begin{tabular}[c]{|cccccc|}\hline
        \multicolumn{1}{|c|}{ } & 待估参数 & 其他参数 & 枢轴量及分布 & 置信区间 & 单侧置信限 \\ \hline
        \multicolumn{1}{|c|}{\multirow{3}{2em}{一个正态总体}} & $\mu$ & $\sigma^2$已知 & & & \\
        \multicolumn{1}{|c|}{ }& $\mu$ & $\sigma^2$未知 & & & \\
        \multicolumn{1}{|c|}{ }& $\sigma^2$ & $\mu$未知 & & & \\ \hline
        \multicolumn{1}{|c|}{\multirow{3}{2em}{两个正态总体}} & $\mu_1-\mu_2$ & $\sigma_1^2,\sigma_2^2$已知 & & & \\
        \multicolumn{1}{|c|}{ }& $\mu_1-\mu_2$ & $\sigma_1^2=\sigma_2^2$未知 & & & \\
        \multicolumn{1}{|c|}{ }& $\frac{\sigma_1^2}{\sigma_2^2}$ & $\mu_1,\mu_2$未知 & & & \\ \hline
    \end{tabular}
    \caption{正态总体均值, 方差的置信区间与单侧置信限}
\end{table*}

注意区分是成对数据还是两正态总体数据
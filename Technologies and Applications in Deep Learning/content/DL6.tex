\newpage
\section{生成模型}
输入: 一堆向量, 表示特征

输出: 图片

什么是好的生成模型? 给定训练集, 产生与训练集同分布的样本. 
%TODO 补图
\begin{align*}
    G^*=\arg \min_G Div(P_G, P_{data})
\end{align*}


\subsection{Variational AutoEncoder(VAE)}
\subsubsection{AutoEncoder(AE)}
图像有冗余, 所以可以降维. 
%TODO 补图
\begin{align*}
    loss=\| x-\hat{x} \|_2
\end{align*}

Why VAE: 填补训练中没有的数据. 

\subsubsection{Denoising AutoEncoder (DAE)}
输入: 有噪音的图像

输出: 图像

但输入输出仍是一一对应的关系. 

\subsubsection{VAE}
%TODO 补图

Encoder 输出 $\mu, \sigma$, 即输出一个预测输入的正态分布. 

重参数化, 从标准正态分布采样, 然后将标准正态分布变为目标正态分布, 以此就可以求得导数. 
\begin{align*}
    N(0,1)\rightarrow N(\mu, \sigma)
\end{align*}

loss 为 KL 散度, 目的是想让Encoder输出的预测更接近标准正态分布($\sigma \rightarrow 1$), 因为纯Encoder会更倾向于$\sigma \rightarrow 0$ (退化为 AE). 
\begin{align*}
    loss=
\end{align*}

\subsubsection{数学推导}

KL散度
\begin{align*}
    D_{KL}(P||Q)=P\log \left(\frac{P}{Q}\right)
\end{align*}
衡量两个分布之间的距离. 

极大似然估计, $\int_z q(z|x)dz=1$, $P(z,x)=P(z|x)P(x)=P(x|z)P(z)$
\begin{align*}
    P(x)&=\int_z P(z)P(x|z)dz\\
    L&=\sum_x\log P(x) \text{ 尽量大}\\
    \log P(x)&=\int_z q(z|x)\log P(x)dz\\
    &=\int_z q(z|x) \log\left( \frac{P(z,x)}{P(z|x)} \right)dz\\
    &=\int_z q(z|x) \log\left( \frac{P(z,x)}{q(z|x)}\frac{q(z|x)}{P(z|x)} \right)dz\\
    &=\int_z q(z|x) \log\left( \frac{P(z,x)}{q(z|x)} \right)dz\\
    &+\underbrace{\int_z q(z|x) \log\left( \frac{q(z|x)}{P(z|x)} \right) dz}_{D_{KL}(q(z|x)||P(z|x))}\\
    &\ge\int_z q(z|x) \log\left( \frac{P(z|x)P(z)}{q(z|x)} \right) dz
\end{align*}
$P(z)$为正态分布, $x|z \sim N(\mu(z), \sigma(z))$, $\mu(z), \sigma(z)$为待估计的参数. 

\begin{align*}
    L_b&=\int_z q(z|x) \log\left( \frac{P(z|x)P(z)}{q(z|x)} \right)dz \\
    &=\underbrace{\int_z q(z|x) \log\left( \frac{P(z)}{q(z|x)} \right) dz}_{-D_{KL}(q(z|x)|| P(z))}+\underbrace{\int_z q(z|x) \log P(x|z) dz}_{\|x-\bar{x} \|_2}\\
    &\le \int_z q(z|x) \log P(x|z) dz
\end{align*}
当$D_{KL}(q(z|x)|| P(z))=0$时取等号. 最大化其下界, 近似最大化其最大值

\subsubsection{总结}


\subsection{GAN}
输入: 取自特定分布的随机噪声

输出: 采样自训练样本分布图像

即从特定分布到图像的映射

生成器与判别器(生成器与VAE的decoder其实是一样的)

\subsubsection{训练}
使用minmax方式联合训练
\begin{align*}
    \min_{\theta_g}\max_{\theta_d}\left[ \mathbb{E}  \right]
\end{align*}
先优化$\theta_d$再优化$\theta_g$. 


交替完成
\begin{enumerate}
    \item G
    \item G
    
    \subitem 但这个函数不是很好

    \subitem 小寄巧:取最大值
    \begin{align*}
        \max E -\log(D(G(z)))
    \end{align*}
\end{enumerate}

\subsubsection{证明}

极大化似然估计等价于最小化KL散度

\begin{align*}
    \theta^*&=
\end{align*}


\subsection{Diffusion model}
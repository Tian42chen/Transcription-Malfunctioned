\newpage
\section{Introduction}

A compiler is a program to translates one language to another.

A Real program language Tiger: Simple and Nontrivial

Two Important Concepts:
\begin{itemize}
    \item Phases(阶段): one or more modules
    \subitem Operating on the different abstract ``languages'' during compiling process
    \item Interfaces(接口)
    \subitem Describe the information exchanged between modules of the compiler
\end{itemize}

\subsection{Modules and Interfaces}

\subsubsection{Modules}
Role: implementing each phase

Advantage: allowing for reuse of the components

\subsubsection{Interfaces}
The data structures: Abstract Syntax, IR Trees and Assem.

A set of functions: The translate interface.

\subsubsection{Phases}
\begin{enumerate}
    \item Lex
    \item Parse
    \item Parsing Actions
    \item Semantic Analysis
    \item Frame Layout
    \item Translate
    \item Canonicalize
    \item Instruction Selection
    \item Control Flow Analysis
    \item Dataflow Analysis
    \item Register Allocation
    \item Code Emission
\end{enumerate}

\subsubsection{Modularization(模块化)}

\subsection{Tools and Software}
Two of the most useful abstractions:
\begin{enumerate}
    \item Context-Free Grammars for parsing
    \item Regular Expressions for lexical analysis
\end{enumerate}

Two tools for compiling:
\begin{enumerate}
    \item Yacc converts a grammar into a parsing program
    \item Lex converts a declarative specification(声明性规范) into a lexical analysis program
\end{enumerate}

\subsection{Data structures for tree languages}

\subsubsection{Intermediate Representations (IR)}
The form of a compiling program: Trees Representation(TR):
\begin{itemize}
    \item The main representation forms
    \item Several node types with different attributes
\end{itemize}

\subsubsection{An example of a program}


\subsubsection{Programming style}
Several conventions for representing tree data structures in C
\begin{enumerate}
    \item Trees are described by a grammar
    \item A tree is described by one or more typedef, each corresponding to a symbol in the grammar.
    \item Each typedef defines a pointer to a corresponding struct.
    \subitem The struct name, which ends in an underscore, is never used anywhere except in the declaration of the typedef and the definition of the struct itself.
    \item Each struct contains a kind fields
    \subitem An enum showing different variants, one of each grammar rule; and a u field, which is a union.
    \item There is more than one nontrivial(value-carraying) symbol in the right-hand side of a rule. The union has a component that is itself a struct comprising these values
    \item %TODO 
\end{enumerate}


\subsubsection{Modularity principle for C programs}
\begin{enumerate}
    \item %TODO 
\end{enumerate}

\newpage
\section{Overview of Semantic Analysis}

\subsection{Abstract Parse Trees}
这种方法限制编译器按照解析程序的顺序来分析程序.

parse tree(解析树) 从 semantic 中分离 issues of syntax(parsing), 并让编译器后期可以遍历它. 
\begin{itemize}
    \item Concrete parse tree: 代表了源码的 concrete syntax. 从技术上讲, 对于输入的每个 token 只有一个 leaf, 对于 parse 中减少的每个语法规则只有一个内部节点. 但其中的标点符号是冗余信息. 
    \item Abstract parse tree/Abstract syntax tree: abstract syntax 表达了源码的 phrase structure, 在没有任何语义解释下解决所有 parsing 问题. 
\end{itemize}
一般是语义分析 concrete syntax 然后构建 abstract syntax tree.

\subsubsection{Positions}
%TODO 摸了, 似乎不考(虽然期中涉及到但没写出来也没扣分)

\subsection{Symbol Table}
\begin{definition}
    A symbol table is a data structure that tracks the current bindings of identifiers
\end{definition}


\begin{code}
    \caption{A Fancier Symbol Table}
    \begin{minted}{c++}
        enter_scope()   // start a new nested scope
        find_symbol(x)  // finds current x (or null)
        add_symbol(x)   // add a symbol x to the table
        check_scope(x)  // true if x defined in current scope
        exit_scope()    // exit current scope
    \end{minted}
\end{code}

局部变量都有一个作用域(scope), 变量仅在自己的作用域中可见. 

环境是由绑定(binding)组成的集合, 指标识符和含义之间的一种映射关系, 用箭头表示. e.g. $\{g\to string\}$, $\{a\to int\}$

\subsubsection{Imperative(命令式)}
\begin{itemize}
    \item 实现: bucket list(hash table)
\end{itemize}
插入 identifiers 时插入到头部, 在退出 scope 时方便 pop. 

\subsubsection{Functional(函数式)}
\begin{itemize}
    \item 实现: 可持久化二叉搜索树(persistent BST)
\end{itemize}
使用可持久化来控制 scope
\newpage
\section{Overview of Semantic Analysis}

\subsection{Symbol Table}
\begin{definition}
    A symbol table is a data structure that tracks the current bindings of identifiers
\end{definition}


\begin{code}
    \caption{A Fancier Symbol Table}
    \begin{minted}{c++}
        enter_scope()   // start a new nested scope
        find_symbol(x)  // finds current x (or null)
        add_symbol(x)   // add a symbol x to the table
        check_scope(x)  // true if x defined in current scope
        exit_scope()    // exit current scope
    \end{minted}
\end{code}

局部变量都有一个作用域(scope), 变量仅在自己的作用域中可见. 

环境是由绑定(binding)组成的集合, 指标识符和含义之间的一种映射关系, 用箭头表示.

\subsubsection{Imperative(命令式)}
\begin{itemize}
    \item 实现: bucket list(hash table)
\end{itemize}
插入 identifiers 时插入到头部, 在退出 scope 时方便 pop. 

\subsubsection{Functional(函数式)}
\begin{itemize}
    \item 实现: 可持久化二叉搜索树(persistent BST)
\end{itemize}
使用可持久化来控制 scope